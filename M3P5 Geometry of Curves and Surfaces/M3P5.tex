\def\module{M3P5 Geometry of Curves and Surfaces}
\def\lecturer{Prof Tom Coates}
\def\term{Autumn 2018}
\def\cover{
$$
\begin{tikzpicture}[scale=0.8]
\draw (-7, 1) to [bend right=60] (-7, -1);
\draw (-7, 1) to [bend left=30] (-4, 1);
\draw (-7, -1) to [bend right=30] (-4, -1);
\draw (-6.2, 0) to [bend right=30] (-4.8, 0);
\draw (-6, -0.1) to [bend left=30] (-5, -0.1);
\draw (-4, 1) to [bend left=30] (-1, 1);
\draw (-4, -1) to [bend right=30] (-1, -1);
\draw (-3.2, 0) to [bend right=30] (-1.8, 0);
\draw (-3, -0.1) to [bend left=30] (-2, -0.1);
\draw [dotted, thick] (-0.5, 0) to (0.5, 0);
\draw (1, 1) to [bend left=30] (4, 1);
\draw (1, -1) to [bend right=30] (4, -1);
\draw (1.8, 0) to [bend right=30] (3.2, 0);
\draw (2, -0.1) to [bend left=30] (3, -0.1);
\draw (4, 1) to [bend left=30] (7, 1);
\draw (4, -1) to [bend right=30] (7, -1);
\draw (4.8, 0) to [bend right=30] (6.2, 0);
\draw (5, -0.1) to [bend left=30] (6, -0.1);
\draw (7, 1) to [bend left=60] (7, -1);
\draw (0, 2.5) node[above]{$ g $};
\draw (-7.5, 1) to [out=45, in=-135] (0, 2.5);
\draw (7.5, 1) to [out=135, in=-45] (0, 2.5);
\end{tikzpicture}
$$
$$ \Downarrow $$
$$
\begin{tikzpicture}[scale=0.8]
\draw (-8.5, 1) to [bend right=60] (-8.5, -1);
\draw (-8.5, 1) to [bend left=30] (-5.5, 1);
\draw (-8.5, -1) to [bend right=30] (-5.5, -1);
\draw (-7.7, 0) to [bend right=30] (-6.3, 0);
\draw (-7.5, -0.1) to [bend left=30] (-6.5, -0.1);
\fill [lightgray] (-5.5, 0) ellipse (0.25 and 1);
\draw [dashed] (-5.5, 1) arc (90:-90:0.25 and 1);
\draw [dashed, thick] (-5.5, 1) arc (90:270:0.25 and 1);
\draw [dotted] (-4.5, 1) arc (90:-90:0.25 and 1);
\draw [dashed, thick] (-4.5, 1) arc (90:270:0.25 and 1);
\draw (-4.5, 1) to [bend left=30] (-1.5, 1);
\draw (-4.5, -1) to [bend right=30] (-1.5, -1);
\draw (-3.7, 0) to [bend right=30] (-2.3, 0);
\draw (-3.5, -0.1) to [bend left=30] (-2.5, -0.1);
\fill [lightgray] (-1.5, 0) ellipse (0.25 and 1);
\draw [dashed] (-1.5, 1) arc (90:-90:0.25 and 1);
\draw [dashed, thick] (-1.5, 1) arc (90:270:0.25 and 1);
\draw [dotted, thick] (-0.5, 0) to (0.5, 0);
\draw [dotted] (1.5, 1) arc (90:-90:0.25 and 1);
\draw [dashed, thick] (1.5, 1) arc (90:270:0.25 and 1);
\draw (1.5, 1) to [bend left=30] (4.5, 1);
\draw (1.5, -1) to [bend right=30] (4.5, -1);
\draw (2.3, 0) to [bend right=30] (3.7, 0);
\draw (2.5, -0.1) to [bend left=30] (3.5, -0.1);
\fill [lightgray] (4.5, 0) ellipse (0.25 and 1);
\draw [dashed] (4.5, 1) arc (90:-90:0.25 and 1);
\draw [dashed, thick] (4.5, 1) arc (90:270:0.25 and 1);
\draw [dotted] (5.5, 1) arc (90:-90:0.25 and 1);
\draw [dashed, thick] (5.5, 1) arc (90:270:0.25 and 1);
\draw (5.5, 1) to [bend left=30] (8.5, 1);
\draw (5.5, -1) to [bend right=30] (8.5, -1);
\draw (6.3, 0) to [bend right=30] (7.7, 0);
\draw (6.5, -0.1) to [bend left=30] (7.5, -0.1);
\draw (8.5, 1) to [bend left=60] (8.5, -1);
\draw (0, 2) node[above]{$ g - 2 $};
\draw (-5, 1) to [out=45, in=-135] (0, 2);
\draw (5, 1) to [out=135, in=-45] (0, 2);
\end{tikzpicture}
$$
$$ \Downarrow $$
$$
\begin{tikzpicture}[scale=0.8]
\fill [lightgray] (-7, 0) to (-6, 0) to (-7, -1) to cycle;
\draw [dashed, thick] (-7, 0) to (-6, 0) to (-7, -1) to cycle;
\fill (-8, 1) circle (0.1);
\fill (-7, 1) circle (0.1);
\draw (-6, 1) circle (0.1);
\fill (-8, 0) circle (0.1);
\fill (-7, 0) circle (0.1);
\draw (-6, 0) circle (0.1);
\draw (-8, -1) circle (0.1);
\draw (-7, -1) circle (0.1);
\draw (-6, -1) circle (0.1);
\draw (-8, 1) to (-6, 1);
\draw (-8, 0) to (-7, 0);
\draw [dotted, thick] (-8, -1) to (-6, -1);
\draw (-8, 1) to (-8, -1);
\draw (-7, 1) to (-7, 0);
\draw [dotted, thick] (-6, 1) to (-6, -1);
\draw (-8, 0) to (-7, 1);
\draw (-8, -1) to (-6, 1);
\draw (-7, -1.5) node{$ \chi\br{\Sigma_{1, 1}} = -1 $};
\fill [lightgray] (-4, 0) to (-3, 1) to (-3, 0) to cycle;
\draw [dotted, thick] (-4, 0) to (-3, 1) to (-3, 0) to cycle;
\fill [lightgray] (-3, 0) to (-2, 0) to (-3, -1) to cycle;
\draw [dashed, thick] (-3, 0) to (-2, 0) to (-3, -1) to cycle;
\draw (-4, 1) circle (0.1);
\draw (-3, 1) circle (0.1);
\draw (-2, 1) circle (0.1);
\draw (-4, 0) circle (0.1);
\fill (-3, 0) circle (0.1);
\draw (-2, 0) circle (0.1);
\draw (-4, -1) circle (0.1);
\draw (-3, -1) circle (0.1);
\draw (-2, -1) circle (0.1);
\draw (-4, 1) to (-2, 1);
\draw [dotted, thick] (-4, -1) to (-2, -1);
\draw (-4, 1) to (-4, -1);
\draw [dotted, thick] (-2, 1) to (-2, -1);
\draw (-4, -1) to (-2, 1);
\draw (-3, -1.5) node{$ \chi\br{\Sigma_{1, 2}} = -2 $};
\draw [dotted, thick] (-0.5, 0) to (0.5, 0);
\fill [lightgray] (2, 0) to (3, 1) to (3, 0) to cycle;
\draw [dotted, thick] (2, 0) to (3, 1) to (3, 0) to cycle;
\fill [lightgray] (3, 0) to (4, 0) to (3, -1) to cycle;
\draw [dashed, thick] (3, 0) to (4, 0) to (3, -1) to cycle;
\draw (2, 1) circle (0.1);
\draw (3, 1) circle (0.1);
\draw (4, 1) circle (0.1);
\draw (2, 0) circle (0.1);
\fill (3, 0) circle (0.1);
\draw (4, 0) circle (0.1);
\draw (2, -1) circle (0.1);
\draw (3, -1) circle (0.1);
\draw (4, -1) circle (0.1);
\draw (2, 1) to (4, 1);
\draw [dotted, thick] (2, -1) to (4, -1);
\draw (2, 1) to (2, -1);
\draw [dotted, thick] (4, 1) to (4, -1);
\draw (2, -1) to (4, 1);
\draw (3, -1.5) node{$ \chi\br{\Sigma_{1, g - 1}} = -2 $};
\fill [lightgray] (6, 0) to (7, 1) to (7, 0) to cycle;
\draw [dotted, thick] (6, 0) to (7, 1) to (7, 0) to cycle;
\draw (6, 1) circle (0.1);
\draw (7, 1) circle (0.1);
\draw (8, 1) circle (0.1);
\draw (6, 0) circle (0.1);
\fill (7, 0) circle (0.1);
\draw (8, 0) circle (0.1);
\draw (6, -1) circle (0.1);
\draw (7, -1) circle (0.1);
\draw (8, -1) circle (0.1);
\draw (6, 1) to (8, 1);
\draw (7, 0) to (8, 0);
\draw [dotted, thick] (6, -1) to (8, -1);
\draw (6, 1) to (6, -1);
\draw (7, 0) to (7, -1);
\draw [dotted, thick] (8, 1) to (8, -1);
\draw (6, -1) to (8, 1);
\draw (7, -1) to (8, 0);
\draw (7, -1.5) node{$ \chi\br{\Sigma_{1, g}} = -1 $};
\draw (0, 2) node[above]{$ g - 2 $};
\draw (-4.5, 1) to [out=45, in=-135] (0, 2);
\draw (4.5, 1) to [out=135, in=-45] (0, 2);
\end{tikzpicture}
$$
$$ \chi\br{\Sigma_g} = \sum_i \chi\br{\Sigma_{1, i}} = \sum_i \br{V_i - E_i + F_i} = 2 - 2g $$
}
\def\syllabus{Parametrisations of curves in three-dimensional space. Curvature. Torsion. Frenet–Serret formulae. Winding number. Charts of surfaces. Tangent vectors. Tangent planes. Smooth maps between surfaces. Normal vectors. The first fundamental form. The second fundamental form. Christoffel symbols. Normal curvature. Gaussian curvature. Mean curvature. Gauss's Theorema Egregium. Area of surfaces. Length-minimising curves. Geodesic curvature. The Gauss–Bonnet theorem and applications.}
\def\thm{subsection}

\documentclass{article}

% Packages

\usepackage{amssymb}
\usepackage{amsthm}
\usepackage[UKenglish]{babel}
\usepackage{commath}
\usepackage{enumitem}
\usepackage{etoolbox}
\usepackage{fancyhdr}
\usepackage[margin=1in]{geometry}
\usepackage{graphicx}
\usepackage[hidelinks]{hyperref}
\usepackage[utf8]{inputenc}
\usepackage{listings}
\usepackage{mathtools}
\usepackage{stmaryrd}
\usepackage{tikz-cd}
\usepackage{csquotes}

% Formatting

\addto\captionsUKenglish{\renewcommand{\abstractname}{Syllabus}}
\delimitershortfall5pt
\ifx\thm\undefined\newtheorem{n}{}\else\newtheorem{n}{}[\thm]\fi
\newcommand\newoperator[1]{\ifcsdef{#1}{\cslet{#1}{\relax}}{}\csdef{#1}{\operatorname{#1}}}
\setlength{\parindent}{0cm}

% Environments

\theoremstyle{plain}
\newtheorem{algorithm}[n]{Algorithm}
\newtheorem*{algorithm*}{Algorithm}
\newtheorem{algorithm**}{Algorithm}
\newtheorem{conjecture}[n]{Conjecture}
\newtheorem*{conjecture*}{Conjecture}
\newtheorem{conjecture**}{Conjecture}
\newtheorem{corollary}[n]{Corollary}
\newtheorem*{corollary*}{Corollary}
\newtheorem{corollary**}{Corollary}
\newtheorem{lemma}[n]{Lemma}
\newtheorem*{lemma*}{Lemma}
\newtheorem{lemma**}{Lemma}
\newtheorem{proposition}[n]{Proposition}
\newtheorem*{proposition*}{Proposition}
\newtheorem{proposition**}{Proposition}
\newtheorem{theorem}[n]{Theorem}
\newtheorem*{theorem*}{Theorem}
\newtheorem{theorem**}{Theorem}

\theoremstyle{definition}
\newtheorem{aim}[n]{Aim}
\newtheorem*{aim*}{Aim}
\newtheorem{aim**}{Aim}
\newtheorem{axiom}[n]{Axiom}
\newtheorem*{axiom*}{Axiom}
\newtheorem{axiom**}{Axiom}
\newtheorem{condition}[n]{Condition}
\newtheorem*{condition*}{Condition}
\newtheorem{condition**}{Condition}
\newtheorem{definition}[n]{Definition}
\newtheorem*{definition*}{Definition}
\newtheorem{definition**}{Definition}
\newtheorem{example}[n]{Example}
\newtheorem*{example*}{Example}
\newtheorem{example**}{Example}
\newtheorem{exercise}[n]{Exercise}
\newtheorem*{exercise*}{Exercise}
\newtheorem{exercise**}{Exercise}
\newtheorem{fact}[n]{Fact}
\newtheorem*{fact*}{Fact}
\newtheorem{fact**}{Fact}
\newtheorem{goal}[n]{Goal}
\newtheorem*{goal*}{Goal}
\newtheorem{goal**}{Goal}
\newtheorem{law}[n]{Law}
\newtheorem*{law*}{Law}
\newtheorem{law**}{Law}
\newtheorem{plan}[n]{Plan}
\newtheorem*{plan*}{Plan}
\newtheorem{plan**}{Plan}
\newtheorem{problem}[n]{Problem}
\newtheorem*{problem*}{Problem}
\newtheorem{problem**}{Problem}
\newtheorem{question}[n]{Question}
\newtheorem*{question*}{Question}
\newtheorem{question**}{Question}
\newtheorem{warning}[n]{Warning}
\newtheorem*{warning*}{Warning}
\newtheorem{warning**}{Warning}
\newtheorem{acknowledgements}[n]{Acknowledgements}
\newtheorem*{acknowledgements*}{Acknowledgements}
\newtheorem{acknowledgements**}{Acknowledgements}
\newtheorem{annotations}[n]{Annotations}
\newtheorem*{annotations*}{Annotations}
\newtheorem{annotations**}{Annotations}
\newtheorem{assumption}[n]{Assumption}
\newtheorem*{assumption*}{Assumption}
\newtheorem{assumption**}{Assumption}
\newtheorem{conclusion}[n]{Conclusion}
\newtheorem*{conclusion*}{Conclusion}
\newtheorem{conclusion**}{Conclusion}
\newtheorem{claim}[n]{Claim}
\newtheorem*{claim*}{Claim}
\newtheorem{claim**}{Claim}
\newtheorem{notation}[n]{Notation}
\newtheorem*{notation*}{Notation}
\newtheorem{notation**}{Notation}
\newtheorem{note}[n]{Note}
\newtheorem*{note*}{Note}
\newtheorem{note**}{Note}
\newtheorem{remark}[n]{Remark}
\newtheorem*{remark*}{Remark}
\newtheorem{remark**}{Remark}

% Lectures

\newcommand{\lecture}[3]{ % Lecture
  \marginpar{
    Lecture #1 \\
    #2 \\
    #3
  }
}

% Blackboard

\renewcommand{\AA}{\mathbb{A}} % Blackboard A
\newcommand{\BB}{\mathbb{B}}   % Blackboard B
\newcommand{\CC}{\mathbb{C}}   % Blackboard C
\newcommand{\DD}{\mathbb{D}}   % Blackboard D
\newcommand{\EE}{\mathbb{E}}   % Blackboard E
\newcommand{\FF}{\mathbb{F}}   % Blackboard F
\newcommand{\GG}{\mathbb{G}}   % Blackboard G
\newcommand{\HH}{\mathbb{H}}   % Blackboard H
\newcommand{\II}{\mathbb{I}}   % Blackboard I
\newcommand{\JJ}{\mathbb{J}}   % Blackboard J
\newcommand{\KK}{\mathbb{K}}   % Blackboard K
\newcommand{\LL}{\mathbb{L}}   % Blackboard L
\newcommand{\MM}{\mathbb{M}}   % Blackboard M
\newcommand{\NN}{\mathbb{N}}   % Blackboard N
\newcommand{\OO}{\mathbb{O}}   % Blackboard O
\newcommand{\PP}{\mathbb{P}}   % Blackboard P
\newcommand{\QQ}{\mathbb{Q}}   % Blackboard Q
\newcommand{\RR}{\mathbb{R}}   % Blackboard R
\renewcommand{\SS}{\mathbb{S}} % Blackboard S
\newcommand{\TT}{\mathbb{T}}   % Blackboard T
\newcommand{\UU}{\mathbb{U}}   % Blackboard U
\newcommand{\VV}{\mathbb{V}}   % Blackboard V
\newcommand{\WW}{\mathbb{W}}   % Blackboard W
\newcommand{\XX}{\mathbb{X}}   % Blackboard X
\newcommand{\YY}{\mathbb{Y}}   % Blackboard Y
\newcommand{\ZZ}{\mathbb{Z}}   % Blackboard Z

% Brackets

\renewcommand{\eval}[1]{\left. #1 \right|}                     % Evaluation
\newcommand{\br}{\del}                                         % Brackets
\newcommand{\abr}[1]{\left\langle #1 \right\rangle}            % Angle brackets
\newcommand{\fbr}[1]{\left\lfloor #1 \right\rfloor}            % Floor brackets
\newcommand{\st}{\ \middle| \ }                                % Such that
\newcommand{\intd}[4]{\int_{#1}^{#2} \, #3 \, \dif #4}         % Single integral
\newcommand{\iintd}[4]{\iint_{#1} \, #2 \, \dif #3 \, \dif #4} % Double integral

% Calligraphic

\newcommand{\AAA}{\mathcal{A}} % Calligraphic A
\newcommand{\BBB}{\mathcal{B}} % Calligraphic B
\newcommand{\CCC}{\mathcal{C}} % Calligraphic C
\newcommand{\DDD}{\mathcal{D}} % Calligraphic D
\newcommand{\EEE}{\mathcal{E}} % Calligraphic E
\newcommand{\FFF}{\mathcal{F}} % Calligraphic F
\newcommand{\GGG}{\mathcal{G}} % Calligraphic G
\newcommand{\HHH}{\mathcal{H}} % Calligraphic H
\newcommand{\III}{\mathcal{I}} % Calligraphic I
\newcommand{\JJJ}{\mathcal{J}} % Calligraphic J
\newcommand{\KKK}{\mathcal{K}} % Calligraphic K
\newcommand{\LLL}{\mathcal{L}} % Calligraphic L
\newcommand{\MMM}{\mathcal{M}} % Calligraphic M
\newcommand{\NNN}{\mathcal{N}} % Calligraphic N
\newcommand{\OOO}{\mathcal{O}} % Calligraphic O
\newcommand{\PPP}{\mathcal{P}} % Calligraphic P
\newcommand{\QQQ}{\mathcal{Q}} % Calligraphic Q
\newcommand{\RRR}{\mathcal{R}} % Calligraphic R
\newcommand{\SSS}{\mathcal{S}} % Calligraphic S
\newcommand{\TTT}{\mathcal{T}} % Calligraphic T
\newcommand{\UUU}{\mathcal{U}} % Calligraphic U
\newcommand{\VVV}{\mathcal{V}} % Calligraphic V
\newcommand{\WWW}{\mathcal{W}} % Calligraphic W
\newcommand{\XXX}{\mathcal{X}} % Calligraphic X
\newcommand{\YYY}{\mathcal{Y}} % Calligraphic Y
\newcommand{\ZZZ}{\mathcal{Z}} % Calligraphic Z

% Fraktur

\newcommand{\aaa}{\mathfrak{a}}   % Fraktur a
\newcommand{\bbb}{\mathfrak{b}}   % Fraktur b
\newcommand{\ccc}{\mathfrak{c}}   % Fraktur c
\newcommand{\ddd}{\mathfrak{d}}   % Fraktur d
\newcommand{\eee}{\mathfrak{e}}   % Fraktur e
\newcommand{\fff}{\mathfrak{f}}   % Fraktur f
\renewcommand{\ggg}{\mathfrak{g}} % Fraktur g
\newcommand{\hhh}{\mathfrak{h}}   % Fraktur h
\newcommand{\iii}{\mathfrak{i}}   % Fraktur i
\newcommand{\jjj}{\mathfrak{j}}   % Fraktur j
\newcommand{\kkk}{\mathfrak{k}}   % Fraktur k
\renewcommand{\lll}{\mathfrak{l}} % Fraktur l
\newcommand{\mmm}{\mathfrak{m}}   % Fraktur m
\newcommand{\nnn}{\mathfrak{n}}   % Fraktur n
\newcommand{\ooo}{\mathfrak{o}}   % Fraktur o
\newcommand{\ppp}{\mathfrak{p}}   % Fraktur p
\newcommand{\qqq}{\mathfrak{q}}   % Fraktur q
\newcommand{\rrr}{\mathfrak{r}}   % Fraktur r
\newcommand{\sss}{\mathfrak{s}}   % Fraktur s
\newcommand{\ttt}{\mathfrak{t}}   % Fraktur t
\newcommand{\uuu}{\mathfrak{u}}   % Fraktur u
\newcommand{\vvv}{\mathfrak{v}}   % Fraktur v
\newcommand{\www}{\mathfrak{w}}   % Fraktur w
\newcommand{\xxx}{\mathfrak{x}}   % Fraktur x
\newcommand{\yyy}{\mathfrak{y}}   % Fraktur y
\newcommand{\zzz}{\mathfrak{z}}   % Fraktur z

% Maps

\newcommand{\bijection}[7][]{    % Bijection
  \ifx &#1&
    \begin{array}{rcl}
      #2 & \longleftrightarrow & #3 \\
      #4 & \longmapsto         & #5 \\
      #6 & \longmapsfrom       & #7
    \end{array}
  \else
    \begin{array}{ccrcl}
      #1 & : & #2 & \longrightarrow & #3 \\
         &   & #4 & \longmapsto     & #5 \\
         &   & #6 & \longmapsfrom   & #7
    \end{array}
  \fi
}
\newcommand{\correspondence}[2]{ % Correspondence
  \cbr{
    \begin{array}{c}
      #1
    \end{array}
  }
  \qquad
  \leftrightsquigarrow
  \qquad
  \cbr{
    \begin{array}{c}
      #2
    \end{array}
  }
}
\newcommand{\function}[5][]{     % Function
  \ifx &#1&
    \begin{array}{rcl}
      #2 & \longrightarrow & #3 \\
      #4 & \longmapsto     & #5
    \end{array}
  \else
    \begin{array}{ccrcl}
      #1 & : & #2 & \longrightarrow & #3 \\
         &   & #4 & \longmapsto     & #5
    \end{array}
  \fi
}
\newcommand{\functions}[7][]{    % Functions
  \ifx &#1&
    \begin{array}{rcl}
      #2 & \longrightarrow & #3 \\
      #4 & \longmapsto     & #5 \\
      #6 & \longmapsto     & #7
    \end{array}
  \else
    \begin{array}{ccrcl}
      #1 & : & #2 & \longrightarrow & #3 \\
         &   & #4 & \longmapsto     & #5 \\
         &   & #6 & \longmapsto     & #7
    \end{array}
  \fi
}

% Matrices

\newcommand{\onebytwo}[2]{      % One by two matrix
  \begin{pmatrix}
    #1 & #2
  \end{pmatrix}
}
\newcommand{\onebythree}[3]{    % One by three matrix
  \begin{pmatrix}
    #1 & #2 & #3
  \end{pmatrix}
}
\newcommand{\twobyone}[2]{      % Two by one matrix
  \begin{pmatrix}
    #1 \\
    #2
  \end{pmatrix}
}
\newcommand{\twobytwo}[4]{      % Two by two matrix
  \begin{pmatrix}
    #1 & #2 \\
    #3 & #4
  \end{pmatrix}
}
\newcommand{\threebyone}[3]{    % Three by one matrix
  \begin{pmatrix}
    #1 \\
    #2 \\
    #3
  \end{pmatrix}
}
\newcommand{\threebythree}[9]{  % Three by three matrix
  \begin{pmatrix}
    #1 & #2 & #3 \\
    #4 & #5 & #6 \\
    #7 & #8 & #9
  \end{pmatrix}
}

% Operators

\newoperator{ab}    % Abelian
\newoperator{AG}    % Affine geometry
\newoperator{alg}   % Algebraic
\newoperator{Ann}   % Annihilator
\newoperator{area}  % Area
\newoperator{Aut}   % Automorphism
\newoperator{BC}    % Bott-Chern
\newoperator{card}  % Cardinality
\newoperator{ch}    % Characteristic
\newoperator{Cl}    % Class
\newoperator{coker} % Cokernel
\newoperator{col}   % Column
\newoperator{Corr}  % Correspondence
\newoperator{diam}  % Diameter
\newoperator{Disc}  % Discriminant
\newoperator{dom}   % Domain
\newoperator{Eig}   % Eigenvalue
\newoperator{Em}    % Embedding
\newoperator{End}   % Endomorphism
\newoperator{Ext}   % Ext
\newoperator{fd}    % Flat dimension
\newoperator{fin}   % Finite
\newoperator{Fix}   % Fixed
\newoperator{Frac}  % Fraction
\newoperator{Frob}  % Frobenius
\newoperator{Fun}   % Function
\newoperator{Gal}   % Galois
\newoperator{gd}    % Global dimension
\newoperator{GL}    % General linear
\newoperator{Ham}   % Hamming
\newoperator{Hom}   % Homomorphism
\newoperator{Homeo} % Homeomorphism
\newoperator{id}    % Identity
\newoperator{Im}    % Imaginary
\newoperator{im}    % Image
\newoperator{Ind}   % Index
\newoperator{ker}   % Kernel
\newoperator{lcm}   % Least common multiple
\newoperator{lgd}   % Left global dimension
\newoperator{Mat}   % Matrix
\newoperator{mult}  % Multiplicity
\newoperator{new}   % New
\newoperator{Nm}    % Norm
\newoperator{old}   % Old
\newoperator{op}    % Opposite
\newoperator{ord}   % Order
\newoperator{Pay}   % Payley
\newoperator{pd}    % Projective dimension
\newoperator{PG}    % Projective geometry
\newoperator{PGL}   % Projective general linear
\newoperator{prim}  % Primitive
\newoperator{PSL}   % Projective special linear
\newoperator{rad}   % Radical
\newoperator{ran}   % Range
\newoperator{Re}    % Real
\newoperator{Res}   % Residue
\newoperator{rgd}   % Right global dimension
\newoperator{rk}    % Rank
\newoperator{row}   % Row
\newoperator{sgn}   % Sign
\newoperator{Sing}  % Singular
\newoperator{SK}    % Skeleton
\newoperator{SL}    % Special linear
\newoperator{SO}    % Special orthogonal
\newoperator{sp}    % Span
\newoperator{Spec}  % Spectrum
\newoperator{srg}   % Strongly regular graph
\newoperator{Stab}  % Stabiliser
\newoperator{Star}  % Star
\newoperator{supp}  % Support
\newoperator{Sym}   % Symmetric
\newoperator{Tor}   % Tor
\newoperator{tors}  % Torsion
\newoperator{Tr}    % Trace
\newoperator{trdeg} % Transcendence degree
\newoperator{wgd}   % Weak global dimension
\newoperator{wt}    % Weight

% Roman

\newcommand{\A}{\mathrm{A}}   % Roman A
\newcommand{\B}{\mathrm{B}}   % Roman B
\newcommand{\C}{\mathrm{C}}   % Roman C
\newcommand{\D}{\mathrm{D}}   % Roman D
\newcommand{\E}{\mathrm{E}}   % Roman E
\newcommand{\F}{\mathrm{F}}   % Roman F
\newcommand{\G}{\mathrm{G}}   % Roman G
\renewcommand{\H}{\mathrm{H}} % Roman H
\newcommand{\I}{\mathrm{I}}   % Roman I
\newcommand{\J}{\mathrm{J}}   % Roman J
\newcommand{\K}{\mathrm{K}}   % Roman K
\renewcommand{\L}{\mathrm{L}} % Roman L
\newcommand{\M}{\mathrm{M}}   % Roman M
\newcommand{\N}{\mathrm{N}}   % Roman N
\renewcommand{\O}{\mathrm{O}} % Roman O
\renewcommand{\P}{\mathrm{P}} % Roman P
\newcommand{\Q}{\mathrm{Q}}   % Roman Q
\newcommand{\R}{\mathrm{R}}   % Roman R
\renewcommand{\S}{\mathrm{S}} % Roman S
\newcommand{\T}{\mathrm{T}}   % Roman T
\newcommand{\U}{\mathrm{U}}   % Roman U
\newcommand{\V}{\mathrm{V}}   % Roman V
\newcommand{\W}{\mathrm{W}}   % Roman W
\newcommand{\X}{\mathrm{X}}   % Roman X
\newcommand{\Y}{\mathrm{Y}}   % Roman Y
\newcommand{\Z}{\mathrm{Z}}   % Roman Z

\renewcommand{\a}{\mathrm{a}} % Roman a
\renewcommand{\b}{\mathrm{b}} % Roman b
\renewcommand{\c}{\mathrm{c}} % Roman c
\renewcommand{\d}{\mathrm{d}} % Roman d
\newcommand{\e}{\mathrm{e}}   % Roman e
\newcommand{\f}{\mathrm{f}}   % Roman f
\newcommand{\g}{\mathrm{g}}   % Roman g
\newcommand{\h}{\mathrm{h}}   % Roman h
\renewcommand{\i}{\mathrm{i}} % Roman i
\renewcommand{\j}{\mathrm{j}} % Roman j
\renewcommand{\k}{\mathrm{k}} % Roman k
\renewcommand{\l}{\mathrm{l}} % Roman l
\newcommand{\m}{\mathrm{m}}   % Roman m
\renewcommand{\n}{\mathrm{n}} % Roman n
\renewcommand{\o}{\mathrm{o}} % Roman o
\newcommand{\p}{\mathrm{p}}   % Roman p
\newcommand{\q}{\mathrm{q}}   % Roman q
\renewcommand{\r}{\mathrm{r}} % Roman r
\newcommand{\s}{\mathrm{s}}   % Roman s
\renewcommand{\t}{\mathrm{t}} % Roman t
\renewcommand{\u}{\mathrm{u}} % Roman u
\renewcommand{\v}{\mathrm{v}} % Roman v
\newcommand{\w}{\mathrm{w}}   % Roman w
\newcommand{\x}{\mathrm{x}}   % Roman x
\newcommand{\y}{\mathrm{y}}   % Roman y
\newcommand{\z}{\mathrm{z}}   % Roman z

% Tikz

\tikzset{
  arrow symbol/.style={"#1" description, allow upside down, auto=false, draw=none, sloped},
  subset/.style={arrow symbol={\subset}},
  cong/.style={arrow symbol={\cong}}
}

% Fancy header

\pagestyle{fancy}
\lhead{\module}
\rhead{\nouppercase{\leftmark}}

% Make title

\title{\module}
\author{Lectured by \lecturer \\ Typed by David Kurniadi Angdinata}
\date{\term}

\begin{document}

\input{../style/cover}

\section{Introduction}

\lecture{1}{Friday}{05/10/18}

A question is what does it mean for a surface to be curved, more curved, less curved, or differently curved? In fact, what does it mean for a curve to be curved? The goal is to answer these questions. Touches on
\begin{itemize}
\item manifolds, smooth shapes, and differential geometry,
\item topology, and
\item Riemannian geometry and general relativity.
\end{itemize}
The following are references.
\begin{itemize}
\item C B\"ar, Elementary differential geometry, 2010
\end{itemize}
A tentative outline of the material that we will cover is as follows. This may change as the term progresses. If so then an updated outline will be given during lectures.
\begin{itemize}
\item Curves in three-dimensional space.
\begin{itemize}
\item Parametrisations.
\item Curvature and torsion, Frenet–Serret formulae.
\item Curves are determined by curvature and torsion.
\item Winding number.
\end{itemize}
\item Surfaces.
\begin{itemize}
\item Charts.
\item Tangent vectors and tangent planes.
\item Smooth maps between surfaces.
\item Normal vectors.
\end{itemize}
\item Curvature.
\begin{itemize}
\item The first and second fundamental forms.
\item Christoffel symbols.
\item Normal curvature, Gaussian curvature, and mean curvature.
\item Gauss's Theorema Egregium.
\end{itemize}
\item Area of surfaces.
\item Geodesics.
\begin{itemize}
\item Length-minimising curves.
\item Existence, non-existence, and examples.
\item Geodesic curvature.
\end{itemize}
\item The Gauss–Bonnet theorem and applications.
\item The topological classification of surfaces.
\item Vector fields and the Poincar\'e–Hopf theorem.
\end{itemize}

\pagebreak

\section{Curves in three-dimensional space}

The ultimate goal is surfaces. First is curves. What does it mean for a curve to be curved? In fact, what is a curve?

\subsection{What is a curve?}

\begin{definition}
The \textbf{$ n $-dimensional Euclidean space} $ \RR^n $ consists of
$$ \br{x_1, \dots, x_n} = \threebyone{x_1}{\vdots}{x_n}, \qquad x_i \in \RR. $$
$ \RR^n $ is a vector space and an inner product space. If $ x = \br{x_1, \dots, x_n} $ and $ y = \br{y_1, \dots, y_n} $ then
$$ x \cdot y = \sum_{i = 1}^n x_iy_i $$
is the \textbf{inner product} of $ x $ and $ y $. The \textbf{length} of $ x \in \RR^n $ is
$$ \abs{x} = \sqrt{x \cdot x} = \sqrt{\sum_{i = 1}^n x_i^2}. $$
\end{definition}

Focus on $ \RR^2 $ and $ \RR^3 $.

\begin{definition}
A \textbf{parametrised curve} in $ \RR^n $ is a smooth map
$$ \phi : \sbr{a, b} \to \RR^n. $$
It is \textbf{regular} if
$$ \phi'\br{t} \ne 0, \qquad t \in \sbr{a, b}. $$
\end{definition}

The curve is $ \phi\br{\sbr{a, b}} $. The parametrised curve says where a particle moving along this curve is positioned at time $ t $. This is $ \phi\br{t} $.

\begin{example*}
$$ \function[\phi]{\sbr{0, 2\pi}}{\RR^2}{t}{\br{\cos t, \sin t}}, \qquad \function[\phi]{\sbr{0, 1}}{\RR^3}{t}{\br{\cos 4\pi t, \sin 4\pi t, t^4}} $$
are curves.
\end{example*}

\begin{example*}
$$ \function[\phi]{\sbr{-1, 1}}{\RR^2}{t}{\br{t, \abs{t}}} $$
is not a curve.
\end{example*}

For us, all parametrised curves will be regular.

\begin{example*}
$$ \function[\phi]{\sbr{-1, 1}}{\RR^2}{t}{\br{t^2, t^3}} $$
is not a curve, since $ \phi'\br{t} = \br{2t, 3t^2} $ and $ \phi'\br{0} = \br{0, 0} $.
\end{example*}

We are not interested in the parametrisation, just the curve.

\begin{example*}
$$ \function[\phi]{\sbr{0, 1}}{\RR^2}{t}{\br{\cos 2\pi t, \sin 2\pi t}} $$
is the curve above.
\end{example*}

\pagebreak

\begin{definition}
Given a regular parametrised curve $ \phi : \sbr{a, b} \to \RR^n $ in $ \RR^n $ and a smooth map $ f : \sbr{c, d} \xrightarrow{\sim} \sbr{a, b} $, with $ f'\br{t} \ne 0 $ for all $ t $, the curve
$$ \varphi = \phi \circ f : \sbr{c, d} \to \RR^n $$
is called a \textbf{reparametrisation} of $ \phi $.
\end{definition}

\begin{proposition}
If $ \phi $ is regular and $ f : \sbr{c, d} \xrightarrow{\sim} \sbr{a, b} $ is smooth with $ f'\br{t} \ne 0 $ for all $ t $, then $ \varphi = \phi \circ f $ is also a regular parametrised curve.
\end{proposition}

\begin{proof}
Chain rule implies that
$$ \varphi'\br{t} = \phi'\br{f\br{t}} \cdot f'\br{t}. $$
Then $ \phi'\br{f\br{t}} $ is never zero because $ \phi $ is regular and $ f'\br{t} $ is never zero by assumption. Thus $ \phi'\br{t} \ne 0 $.
\end{proof}

So will study parametrised curves up to reparametrisation and study reparametrisation-invariant properties of parametrised curves.

\begin{remark*}
Reparametrisations $ f : \sbr{c, d} \xrightarrow{\sim} \sbr{a, b} $ with $ f'\br{t} \ne 0 $ form a groupoid under composition of functions. The groupoid acts on parametrised curves.
\end{remark*}

\lecture{2}{Monday}{08/10/18}

Say that $ \phi \sim \varphi $ if and only if $ \phi $ is a reparametrisation of $ \varphi $. Then $ \sim $ is an equivalence relation. A \textbf{curve} is an equivalence class of parametrised curves. The \textbf{trace} of a curve $ \phi : \sbr{a, b} \to \RR^n $ is $ \phi\br{\sbr{a, b}} $, that is the image in $ \RR^n $. If $ \phi \sim \varphi $ then the trace of $ \phi $ is equal to the trace of $ \varphi $, so this is well-defined.

\begin{example*}
The tangent line to a curve at a point.
\end{example*}

\begin{definition}
The \textbf{tangent line} $ L $ to $ \phi $ at $ \phi\br{t_0} $ is
$$ L = \cbr{\phi\br{t_0} + s\phi'\br{t_0} \st s \in \RR}. $$
\end{definition}

Check that this is a reparametrisation-invariant. Suppose $ f : \sbr{c, d} \xrightarrow{\sim} \sbr{a, b} $ is a reparametrisation and $ \varphi = \phi \circ f $. Let $ s_0 \in \sbr{c, d} $ satisfy $ f\br{s_0} = t_0 $. Then
$$ \varphi\br{s_0} = \phi\br{f\br{s_0}} = \phi\br{t_0}, \qquad \varphi'\br{s_0} = \phi'\br{f\br{s_0}} \cdot f'\br{s_0} = \phi'\br{t_0} \cdot f'\br{s_0}, $$
so
$$ L = \cbr{\varphi\br{s_0} + s'\varphi'\br{s_0} \st s' \in \RR}. $$

\begin{definition}
The \textbf{length} of a parametrised curve $ \phi : \sbr{a, b} \to \RR^n $ is
$$ \L\br{\phi} = \intd{a}{b}{\abs{\phi'\br{t}}}{t}. $$
\end{definition}

Chopping up $ \sbr{a, b} $ into $ N $ intervals of size $ \Delta t = \br{b - a} / N $,
$$ \L\br{\phi} \approx \sum_{i = 0}^N \abs{\phi'\br{a + i\Delta t}}\Delta t. $$

\begin{example*}
Let
$$ \function[\phi]{\sbr{0, 2\pi}}{\RR^2}{t}{\br{\cos t, \sin t}}. $$
Then $ \phi'\br{t} = \br{-\sin t, \cos t} $, so $ \abs{\phi'\br{t}} = \sqrt{\sin^2 t + \cos^2 t} = 1 $. Thus
$$ \L\br{\phi} = \intd{0}{2\pi}{1}{t} = 2\pi, $$
so the length is $ 2\pi $.
\end{example*}

\pagebreak

\begin{proposition}
$ \L\br{\phi} $ is invariant under reparametrisation.
\end{proposition}

\begin{proof}
For simplicity, suppose $ n = 3 $, so $ \RR^3 $. Then
$$ \function[\phi]{\sbr{a, b}}{\RR^3}{t}{\br{x\br{t}, y\br{t}, z\br{t}}}. $$
Thus
$$ \L\br{\phi} = \intd{a}{b}{\sqrt{\br{\tod{x}{t}}^2 + \br{\tod{y}{t}}^2 + \br{\tod{z}{t}}^2}}{t}. $$
Given a reparametrisation $ f : \sbr{c, b} \xrightarrow{\sim} \sbr{a, b} $, so $ f $ is smooth and $ f'\br{t} \ne 0 $ for all $ t $, set $ \varphi = \phi \circ f $ and write
$$ \varphi\br{s} = \br{X\br{t}, Y\br{t}, Z\br{t}}. $$
Then
$$ \L\br{\varphi} = \intd{c}{d}{\sqrt{\br{\tod{X}{s}}^2 + \br{\tod{Y}{s}}^2 + \br{\tod{Z}{s}}^2}}{s}. $$
Thus use the change of variable formula and
$$ \dod{X}{s}\br{s} = \dod{x}{t}\br{f\br{s}}\dod{f}{s}\br{s}, $$
by the chain rule.
\end{proof}

If $ \abs{\phi'\br{t}} = 1 $ for all $ t $ then
$$ \L\br{\phi} = \intd{a}{b}{\abs{\phi'\br{t}}}{t} = b - a, $$
and in fact $ \L\br{\eval{\phi}_{\sbr{a, t}}} = t - a $, so the curve is parametrised by arc-length. This is an \textbf{arc-length parametrisation} or a \textbf{unit speed parametrisation}.

\begin{proposition}
Let $ \phi : \sbr{a, b} \to \RR^n $ be a regular curve. Then there exists an arc-length parametrisation of $ \phi $.
\end{proposition}

\begin{proof}
Let
$$ \function[l]{\sbr{a, b}}{\sbr{0, L}}{t}{\intd{0}{t}{\abs{\phi'\br{s}}}{s}}. $$
Set $ f = l^{-1} $ as the inverse function. Then if $ \varphi = \phi \circ f $ we have
$$ \abs{\varphi'\br{t}} = \abs{\phi'\br{f\br{t}}}f'\br{t}. $$
Also $ l\br{f\br{t}} = t $, so $ l'\br{f\br{t}}f'\br{t} = 1 $. Thus
$$ f'\br{t} = \dfrac{1}{l'\br{f\br{t}}} = \dfrac{1}{\abs{\phi'\br{f\br{t}}}}, $$
by the fundamental theorem of calculus, so we have $ \abs{\varphi'\br{t}} = 1 $ as required.
\end{proof}

\lecture{3}{Tuesday}{09/10/18}

Is an arc-length parametrisation unique? No. Suppose that $ \phi : \sbr{a, b} \to \RR^n $ is an arc-length parametrisation of a curve, and that $ f : \sbr{c, d} \xrightarrow{\sim} \sbr{a, b} $ is a reparametrisation such that $ \varphi = \phi \circ f $ is also an arc-length parametrisation. Then $ \abs{\phi'\br{t}} = 1 $ for all $ t \in \sbr{a, b} $ and $ \abs{\varphi'\br{t}} = 1 $ for all $ t \in \sbr{c, d} $, so
$$ \abs{\phi'\br{f\br{t}}}\abs{f'\br{t}} = 1, \qquad t \in \sbr{c, d}, $$
so $ \abs{f'\br{t}} = 1 $. Thus $ f'\br{t} = \pm 1 $, so $ f\br{t} = \pm t + C $ for some constant $ C $.

\pagebreak

\subsection{Curvature}

\begin{definition}
Let $ \phi : \sbr{a, b} \to \RR^n $ be a curve parametrised by arc-length. The \textbf{curvature} of $ \phi $ at $ \phi\br{t} $ is
$$ \kappa\br{t} = \abs{\phi''\br{t}}. $$
The \textbf{curvature vector} is
$$ \vec{\kappa}\br{t} = \phi''\br{t}. $$
\end{definition}

Claim that this depends only on the curve not on the parametrisation $ \phi $. If $ \varphi = \phi \circ f $ is another arc-length parametrisation then $ f\br{t} = \pm t + C $. Putting $ s = f\br{t} $ we have
$$ \dod{\varphi}{s} = \pm\dod{\phi}{t}, \qquad \dod[2]{\varphi}{s} = \dod[2]{\phi}{t}, $$
so $ \vec{\kappa}\br{t} $ is independent of the choice of parametrisation.

\begin{proposition}
$ \kappa\br{t} = 0 $ if and only if the curve is a straight line.
\end{proposition}

\begin{proof}
Let $ \phi : \sbr{a, b} \to \RR^n $ be an arc-length parametrisation of the curve. Then $ \kappa\br{t} = 0 $, so $ \phi''\br{t} = 0 $. Thus
$$ \phi\br{t} = \vec{a} + \vec{b}t, \qquad \vec{a}, \vec{b} \in \RR^n. $$
\end{proof}

\begin{proposition}
For curves in $ \RR^n $, the vector curvature $ \vec{\kappa}\br{t} $ is perpendicular to the tangent line at $ \phi\br{t} $ for all $ t $, where $ \phi $ is an arc-length parametrisation.
\end{proposition}

\begin{proof}
The tangent line to the curve at $ \phi\br{t} $ is
$$ L = \cbr{\phi\br{t} + s\phi'\br{t} \st s \in \RR}, $$
where $ \phi'\br{t} $ is the direction vector of the tangent line. Need to show $ \phi'\br{t} \cdot \phi''\br{t} = 0 $ for all $ t $. Know $ \abs{\phi'\br{t}} = 1 $ for all $ t $, that is $ \phi'\br{t} \cdot \phi'\br{t} = 1 $ for all $ t $. Differentiating,
$$ \phi''\br{t} \cdot \phi'\br{t} + \phi'\br{t} \cdot \phi''\br{t} = 0, $$
so $ 2\phi'\br{t} \cdot \phi''\br{t} = 0 $. Thus $ \phi'\br{t} \cdot \phi''\br{t} = 0 $.
\end{proof}

\begin{example*}
The curvature of a circle in $ \RR^2 $ centred at the origin of radius $ R $. Need an arc-length parametrisation. Try
$$ \function[\phi]{\sbr{0, 2\pi}}{\RR}{t}{\br{R\cos t, R\sin t}}. $$
Then
$$ \abs{\phi'\br{t}} = \abs{\br{-R\sin t, R\cos t}} = R. $$
Oops. Try
$$ \function[\phi]{\sbr{0, 2\pi R}}{\RR}{t}{\br{R\cos \tfrac{t}{R}, R\sin \tfrac{t}{R}}}. $$
Checking,
$$ \abs{\phi'\br{t}} = \abs{\br{-\sin \tfrac{t}{R}, \cos \tfrac{t}{R}}} = 1. $$
The vector curvature is
$$ \vec{\kappa}\br{t} = \br{-\tfrac{1}{R}\cos \tfrac{t}{R}, -\tfrac{1}{R}\sin \tfrac{t}{R}}, $$
so the curvature is
$$ \kappa\br{t} = \abs{\vec{\kappa}\br{t}} = \dfrac{1}{R}. $$
\end{example*}

The curvature $ \kappa\br{t} $ does not determine the curve.

\pagebreak

\subsection{Space curves}

Let $ \phi : \sbr{a, b} \to \RR^3 $ be an arc-length parametrisation of a curve. Set
$$ \T\br{t} = \phi'\br{t}, $$
the \textbf{unit tangent vector} to the curve at $ \phi\br{t} $ and assume $ \T'\br{t} \ne 0 $ for all $ t $. Set
$$ \N\br{t} = \dfrac{\T'\br{t}}{\abs{\T'\br{t}}}, $$
the \textbf{principal normal vector} to the curve at $ \phi\br{t} $. Set
$$ \B\br{t} = \T\br{t} \times \N\br{t}, $$
the \textbf{binormal vector} to the curve at $ \phi\br{t} $. Check that $ \abs{\B\br{t}} = 1 $. Thus $ \br{\T, \N, \B} $ is a positively oriented orthonormal basis for $ \RR^3 $, for each $ t \in \sbr{a, b} $. This is the \textbf{Frenet frame} or \textbf{Serret-Frenet frame}.

\lecture{4}{Friday}{12/10/18}

$$ \T'\br{t} = \abs{\phi''\br{t}}\N\br{t} = \kappa\br{t}\N\br{t}. $$
$ \B'\br{t} = \T'\br{t} \times \N\br{t} + \T\br{t} \times \N'\br{t} = \T\br{t} \times \N'\br{t} $, so $ \B'\br{t} $ is perpendicular to $ \T\br{t} $. Then $ \B\br{t} \cdot \B\br{t} = 1 $ for all $ t $, so $ \B\br{t} $ is perpendicular to $ \B'\br{t} $. Thus
$$ \B'\br{t} = -\tau\br{t}\N\br{t}, $$
for some function $ \tau\br{t} $. Then $ \tau\br{t} $ is called the \textbf{torsion} of $ \phi $ at the point $ \phi\br{t} $. Then $ \N\br{t} = \B\br{t} \times \T\br{t} $ so
$$ \N'\br{t} = \B'\br{t} \times \T\br{t} + \B\br{t} \times \T'\br{t} = -\tau\br{t}\N\br{t} \times \T\br{t} + \B\br{t} \times \kappa\br{t}\N\br{t} = \tau\br{t}\B\br{t} - \kappa\br{t}\T\br{t}. $$
Thus
$$ \T'\br{t} = \kappa\br{t} \cdot \N\br{t}, \qquad \B'\br{t} = -\tau\br{t} \cdot \N\br{t}, \qquad \N'\br{t} = \tau\br{t} \cdot \B\br{t} - \kappa\br{t} \cdot \T\br{t}. $$
So the time evolution of the frame depends only on the curvature, the torsion, and the frame itself.

\begin{proposition}
Let $ \phi : \sbr{a, b} \to \RR^3 $ be a curve parametrised by arc-length. Suppose $ \phi''\br{t} \ne 0 $ for all $ t \in \sbr{a, b} $. Need this for the Frenet frame to exist. Then $ \tau\br{t} = 0 $ for all $ t $ if and only if $ \phi $ is planar.
\end{proposition}

\begin{proof}
\hfill
\begin{itemize}
\item[$ \implies $] Suppose $ \tau\br{t} = 0 $. Then $ \B'\br{t} = 0 $, so $ \B\br{t} = v $ is constant. Claim that $ \phi\br{t} \cdot v $ is constant for all $ t $.
$$ \br{\phi\br{t} \cdot v}' = \phi'\br{t} \cdot v + \phi\br{t} \cdot 0 = \phi'\br{t} \cdot v = \T\br{t} \cdot v = 0, $$
since $ \T $ is perpendicular to $ \B $. Thus $ \phi $ is moving in a plane perpendicular to $ v $.
\item[$ \impliedby $] Suppose $ \phi $ is planar. Then $ \phi\br{t} \cdot v = a $ for some constant $ v \in \RR^3 $ and constant $ a \in \RR $. Then $ \phi'\br{t} \cdot v = 0 $, so $ \T\br{t} $ is perpendicular to $ v $. Then $ \phi''\br{t} \cdot v = 0 $, so $ \kappa\br{t} \cdot \N\br{t} $ is perpendicular to $ v $. Then $ \kappa\br{t} \ne 0 $, so $ \N\br{t} $ is perpendicular to $ v $. Thus $ \B\br{t} = \pm v / \abs{v} $ is constant, so $ \tau\br{t} = 0 $.
\end{itemize}
\end{proof}

\begin{exercise*}
The unit circle in the $ \br{x, y} $ plane is
$$ \phi\br{t} = \br{\cos t, \sin t, 0}, \qquad t \in \sbr{0, 2\pi}. $$
Compute $ \T\br{t}, \N\br{t}, \B\br{t}, \tau\br{t}, \kappa\br{t} $. Verify the Frenet formulae.
\end{exercise*}

\pagebreak

\begin{example*}
The helix is
$$ \phi\br{t} = \br{\cos \tfrac{t}{\sqrt{2}}, \sin \tfrac{t}{\sqrt{2}}, \tfrac{t}{\sqrt{2}}}, \qquad t \in \RR. $$
It is an arc-length parametrisation, since $ \phi'\br{t} = \br{-\tfrac{1}{\sqrt{2}}\sin \tfrac{t}{\sqrt{2}}, \tfrac{1}{\sqrt{2}}\cos \tfrac{t}{\sqrt{2}}, \tfrac{1}{\sqrt{2}}} $, so $ \abs{\phi'\br{t}} = 1 $. Then
\begin{align*}
\T\br{t} & = \phi'\br{t} = \br{-\tfrac{1}{\sqrt{2}}\sin \tfrac{t}{\sqrt{2}}, \tfrac{1}{\sqrt{2}}\cos \tfrac{t}{\sqrt{2}}, \tfrac{1}{\sqrt{2}}}, \\
\N\br{t} & = \dfrac{\phi''\br{t}}{\abs{\phi''\br{t}}} = 2\br{-\tfrac{1}{2}\cos \tfrac{t}{\sqrt{2}}, -\tfrac{1}{2}\sin \tfrac{t}{\sqrt{2}}, 0} = \br{-\cos \tfrac{t}{\sqrt{2}}, -\sin \tfrac{t}{\sqrt{2}}, 0}, \\
\B\br{t} & = \T\br{t} \times \N\br{t} = \br{\tfrac{1}{\sqrt{2}}\sin \tfrac{t}{\sqrt{2}}, -\tfrac{1}{\sqrt{2}}\cos \tfrac{t}{\sqrt{2}}, \tfrac{1}{\sqrt{2}}}.
\end{align*}
Then $ \kappa\br{t} = \abs{\phi''\br{t}} = \tfrac{1}{2} $ and $ \tau\br{t} = \tfrac{1}{2} $, since $ \B'\br{t} = \br{\tfrac{1}{2}\cos \tfrac{t}{\sqrt{2}}, \tfrac{1}{2}\sin \tfrac{t}{\sqrt{2}}, 0} = -\tau\br{t} \cdot \N\br{t} $.
\end{example*}

In general $ \kappa $ and $ \tau $ are not constants.

\lecture{5}{Monday}{15/10/18}

\begin{theorem}[Fundamental theorem of the local theory of curves]
Given two differentiable functions $ \kappa : \sbr{a, b} \to \RR $ and $ \tau : \sbr{a, b} \to \RR $ such that $ \kappa\br{t} > 0 $ for all $ t $, there exists a regular curve $ \phi : \sbr{a, b} \to \RR^3 $ parametrised by arc-length with curvature $ \kappa\br{t} $ and torsion $ \tau\br{t} $. Any other such curve $ \varphi : \sbr{a, b} \to \RR^3 $ differs from $ \phi $ by a \textbf{rigid motion}, that is
$$ \varphi = g \cdot \phi + \vec{c}, \qquad g \in \SO\br{3}, \qquad \vec{c} \in \RR^3. $$
\end{theorem}

\begin{proof}
\hfill
\begin{itemize}
\item Proof of uniqueness. Rigid motions preserve arc-length, curvature, and torsion. If $ \widetilde{\phi} = g \cdot \phi + \vec{c} $ then
\begin{itemize}
\item $ \abs{\widetilde{\phi}'} = \abs{g \cdot \phi'} = \abs{\phi'} = 1 $,
\item $ \kappa_{\widetilde{\phi}} = \abs{\widetilde{\phi}''} = \abs{g \cdot \phi''} = \abs{\phi''} = \kappa_\phi $, and
\item $ \T_{\widetilde{\phi}} = \widetilde{\phi}' = g \cdot \phi' = g \cdot \T_\phi $ and $ \N_{\widetilde{\phi}} = \T_{\widetilde{\phi}}' / \abs{\T_{\widetilde{\phi}}'} = g \cdot \T_\phi' / \abs{\T_\phi'} = g \cdot \N_\phi $, so
$$ \B_{\widetilde{\phi}} = \T_{\widetilde{\phi}} \times \N_{\widetilde{\phi}} = g \cdot \T_\phi \times g \cdot \N_\phi = g \cdot \br{\T_\phi \times \N_\phi} = g \cdot \B_\phi, $$
so
$$ -\tau_{\widetilde{\phi}} \cdot \N_{\widetilde{\phi}} = \B_{\widetilde{\phi}}' = \br{g \cdot \B_\phi}' = g \cdot \B_\phi' = g \cdot \br{-\tau_\phi \cdot \N_\phi} = -\tau_\phi \cdot \N_{\widetilde{\phi}}. $$
Thus $ \tau_{\widetilde{\phi}} = \tau_\phi $.
\end{itemize}
We can apply a rigid motion to $ \varphi $ which sends $ \varphi\br{a} $ to $ \phi\br{a} $ and Frenet frame $ \br{\T_\varphi\br{a}, \N_\varphi\br{a}, \B_\varphi\br{a}} $ to $ \br{\T_\phi\br{a}, \N_\phi\br{a}, \B_\phi\br{a}} $. We can assume that these are equal without loss of generality. We compute
\begin{align*}
& \dfrac{1}{2}\dod{}{t}\br{\abr{\T_\phi - \T_\varphi, \T_\phi - \T_\varphi} + \abr{\N_\phi - \N_\varphi, \N_\phi - \N_\varphi} + \abr{\B_\phi - \B_\varphi, \B_\phi - \B_\varphi}} \\
= \ & \abr{\T_\phi - \T_\varphi, \T_\phi' - \T_\varphi'} + \abr{\N_\phi - \N_\varphi, \N_\phi' - \N_\varphi'} + \abr{\B_\phi - \B_\varphi, \B_\phi' - \B_\varphi'} \\
= \ & \kappa\abr{\T_\phi - \T_\varphi, \N_\phi - \N_\varphi} + \br{\tau\abr{\N_\phi - \N_\varphi, \B_\phi - \B_\varphi} - \kappa\abr{\N_\phi - \N_\varphi, \T_\phi - \T_\varphi}} - \tau\abr{\B_\phi - \B_\varphi, \N_\phi - \N_\varphi} \\
= \ & 0,
\end{align*}
by the Frenet equations, and $ \phi $ and $ \varphi $ are solutions to the problem in the statement. Then
$$ \abr{\T_\phi - \T_\varphi, \T_\phi - \T_\varphi} + \abr{\N_\phi - \N_\varphi, \N_\phi - \N_\varphi} + \abr{\B_\phi - \B_\varphi, \B_\phi - \B_\varphi} $$
is constant in time and is zero at $ t = a $, so it is zero for all $ t \in \sbr{a, b} $, so
$$ \T_\phi = \T_\varphi, \qquad \N_\phi = \N_\varphi, \qquad \B_\phi = \B_\varphi. $$
In particular
$$ \phi\br{t} = \phi\br{a} + \intd{a}{t}{\T_\phi\br{s}}{s} = \varphi\br{a} + \intd{a}{t}{\T_\varphi\br{s}}{s} = \varphi\br{t}. $$
The conclusion is $ \phi = \varphi $.

\pagebreak

\item Proof of existence. Given $ \kappa\br{t} > 0 $ and $ \tau\br{t} $, we can pick any positive orthonormal frame $ \br{\T_a, \N_a, \B_a} $ and use the existence theorem for solutions of linear differential equations to find $ \T, \N, \B : \sbr{a, b} \to \RR^3 $ such that
$$
\begin{cases}
\threebyone{\T\br{a}}{\N\br{a}}{\B\br{a}} = \threebyone{\T_a}{\N_a}{\B_a} \\
\threebyone{\T'}{\N'}{\B'} = \threebythree{0}{\kappa}{0}{-\kappa}{0}{\tau}{0}{-\tau}{0}\threebyone{\T}{\N}{\B}
\end{cases}.
$$
We check that $ \br{\T, \N, \B} $ is an orthonormal frame at all $ t \in \sbr{a, b} $. This is true at $ t = a $. We consider
$$ M = \onebythree{\T}{\N}{\B}, \qquad M' = \onebythree{\T'}{\N'}{\B'} = \onebythree{\T}{\N}{\B}\threebythree{0}{-\kappa}{0}{\kappa}{0}{-\tau}{0}{\tau}{0} = M\threebythree{0}{-\kappa}{0}{\kappa}{0}{-\tau}{0}{\tau}{0}. $$
As $ M $ is orthonormal if and only if $ M^\intercal \cdot M = \id $, we want to prove that $ M^\intercal \cdot M = \id $, so $ M^\intercal \cdot M = \id $ at $ t = a $, and
$$ \dod{}{t}\br{M^\intercal \cdot M} = M'^\intercal \cdot M + M^\intercal \cdot M' = \threebythree{0}{\kappa}{0}{-\kappa}{0}{\tau}{0}{-\tau}{0} \cdot M^\intercal \cdot M + M^\intercal \cdot M \cdot \threebythree{0}{-\kappa}{0}{\kappa}{0}{-\tau}{0}{\tau}{0}. $$
The linear system
$$
\begin{cases}
\dod{}{t}\br{A} = \threebythree{0}{\kappa}{0}{-\kappa}{0}{\tau}{0}{-\tau}{0}A + A\threebythree{0}{-\kappa}{0}{\kappa}{0}{-\tau}{0}{\tau}{0} \\
A\br{a} = \id
\end{cases}
$$
has solution $ A\br{t} = \id $. Hence by uniqueness, $ A\br{t} = \id = M^\intercal \cdot M $.
\end{itemize}
\end{proof}

\begin{exercise*}
Why is $ \br{\T\br{t}, \N\br{t}, \B\br{t}} $ orthonormal for all $ t $ equivalent to the skew-symmetry of
$$ \threebythree{0}{\kappa}{0}{-\kappa}{0}{\tau}{0}{-\tau}{0}? $$
\end{exercise*}

\lecture{6}{Tuesday}{16/10/18}

\begin{example*}
Any curve $ \phi : \sbr{a, b} \to \RR^3 $ with zero torsion and constant curvature $ c > 0 $ is an arc of a circle of radius $ 1 / c $. An arc of a circle has this property, since
$$ \phi\br{t} = \br{R\cos \tfrac{t}{R}, R\sin \tfrac{t}{R}, 0}, $$
$$ \phi'\br{t} = \br{-\sin \tfrac{t}{R}, \cos \tfrac{t}{R}, 0}, $$
$$ \phi''\br{t} = \br{-\tfrac{1}{R}\cos \tfrac{t}{R}, -\tfrac{1}{R}\sin \tfrac{t}{R}, 0}, $$
so $ \kappa\br{t} = 1 / R = c $ and $ \tau\br{t} = 0 $, since the circle is a planar curve. Then apply the fundamental theorem of the local theory of curves.
\end{example*}

\pagebreak

\subsection{Plane curves}

Let $ \phi : \sbr{a, b} \to \RR^3 $. Then $ \phi'\br{t} = 0 $ is a point, $ \kappa\br{t} = 0 $ is a line, and $ \tau\br{t} = 0 $ is a plane. Let
$$ \function[\phi]{\sbr{a, b}}{\RR^2}{t}{\br{x\br{t}, y\br{t}}} $$
be parametrised by arc-length, so $ \phi'\br{t} = \br{x'\br{t}, y'\br{t}} = 1 $ and $ \N\br{t} = \br{-y'\br{t}, x'\br{t}} $. Then $ \phi''\br{t} = \kappa\br{t}\N\br{t} $, so
\begin{align*}
\kappa\br{t}
& = \kappa\br{t}\abr{\N\br{t}, \N\br{t}}
= \abr{\kappa\br{t}\N\br{t}, \N\br{t}}
= \abr{\phi''\br{t}, \N\br{t}} \\
& = \abr{\br{x''\br{t}, y''\br{t}}, \br{-y'\br{t}, x'\br{t}}}
= x'\br{t}y''\br{t} - y'\br{t}x''\br{t}.
\end{align*}

\begin{proposition}
For arbitrary $ \phi $,
$$ \kappa\br{t} = \dfrac{\phi''\br{t} \cdot \N\br{t}}{\abs{\phi'\br{t}}^2}. $$
\end{proposition}

\begin{proof}
Let $ \varphi\br{s} = \phi\br{f\br{s}} = \br{x\br{f\br{s}}, y\br{f\br{s}}} $ be a reparametrisation by arc-length. Then
\begin{align*}
\kappa_\varphi\br{s}
& = \br{x\br{f\br{s}}}'\br{y\br{f\br{s}}}'' - y\br{f\br{s}}'x\br{f\br{s}}''
= \br{x'\br{f\br{s}}y''\br{f\br{s}} - y'\br{f\br{s}}x''\br{f\br{s}}} \cdot \br{f'\br{s}}^3 \\
& = \phi''\br{f\br{s}} \cdot \br{-\br{y\br{f\br{s}}}', \br{x\br{f\br{s}}}'} \cdot \br{f'\br{s}}^2
= \br{\phi''\br{f\br{s}} \cdot \N_\varphi\br{f\br{s}}} \cdot \br{f'\br{s}}^2.
\end{align*}
We have $ f = l^{-1} $ and $ l\br{t} = \intd{a}{t}{\abs{\phi'\br{s}}}{s} $, so in particular $ f'\br{s} = 1 / \abs{\phi'\br{f\br{s}}} $. Hence replace $ f\br{s} = t $.
\end{proof}

\begin{example*}
Let
$$ \function[\phi]{\sbr{a, b}}{\RR^2}{t}{\br{t, f\br{t}}} $$
be a graph of a smooth function $ f $. Then
$$ \phi'\br{t} = \br{1, f'\br{t}}, \qquad \phi''\br{t} = \br{0, f''\br{t}}, \qquad \abs{\phi'\br{t}} = 1 + \abs{f'\br{t}}^2, \qquad \N\br{t} = \dfrac{\br{-f'\br{t}, 1}}{\sqrt{1 + \abs{f'\br{t}}^2}}, $$
so
$$ \kappa\br{t} = \dfrac{f''\br{t}}{\sqrt{1 + \br{f'\br{t}}^2}^3}. $$
\end{example*}

Suppose that $ \phi : \sbr{a, b} \to \RR^2 \cong \CC $ is a smooth closed curve, so $ \phi\br{a} = \phi\br{b} $ and $ \phi^{\br{k}}\br{a} = \phi^{\br{k}}\br{b} $ for all $ k $. The \textbf{winding number} of $ \phi $ is an invariant measuring the number of clockwise rotations of the curve. Let $ z = x + iy $. We know from complex analysis
\begin{align*}
\w\br{\phi}
& = \dfrac{1}{2\pi i} \oint_\gamma \, \dfrac{1}{z} \, \d z
= \dfrac{1}{2\pi i} \intd{a}{b}{\dfrac{1}{x\br{t} + iy\br{t}}}{\br{x\br{t} + iy\br{t}}} \\
& = \dfrac{1}{2\pi i} \intd{a}{b}{\dfrac{x'\br{t} + iy'\br{t}}{x\br{t} + iy\br{t}} \cdot \dfrac{x\br{t} - iy\br{t}}{x\br{t} - iy\br{t}}}{t}
= \dfrac{1}{2\pi i} \intd{a}{b}{\dfrac{\br{xx' + yy'} + i\br{y'x - x'y}}{x^2 + y^2}}{t} \\
& = \dfrac{1}{4\pi i} \sbr{\ln\br{x^2 + y^2}}_a^b + \dfrac{1}{2\pi} \intd{a}{b}{\dfrac{y'x - x'y}{x^2 + y^2}}{t}
= \dfrac{1}{2\pi} \intd{a}{b}{\dfrac{y'x - x'y}{x^2 + y^2}}{t},
\end{align*}
since $ \phi $ is a closed curve. If $ \phi $ is parametrised by arc-length, so $ \abs{\phi'} = 1 $, then
$$ \w\br{\phi'} = \dfrac{1}{2\pi} \intd{a}{b}{\kappa\br{t}}{t}. $$

\begin{proposition}
The winding number $ \w\br{\T\br{t}} $ of the unit tangent vector $ \T\br{t} = \phi'\br{t} / \abs{\phi'\br{t}} $ is equal to the \textbf{total curvature} $ \intd{a}{b}{\kappa\br{t}}{t} $, divided by $ 2\pi $.
\end{proposition}

\begin{definition}
The winding number $ \w\br{\T\br{t}} $ is called the \textbf{index} or \textbf{turning number} of $ \phi $, and it is denoted $ \Ind \phi $. If $ \phi'\br{t} = e^{i\theta t} $, then $ \kappa\br{t} = \dot{\theta}\br{t} $. Thus
$$ \Ind \phi = \dfrac{1}{2\pi}\intd{a}{b}{\dot{\theta}\br{t}}{t}. $$
\end{definition}


\pagebreak

\section{Surfaces}

\lecture{7}{Friday}{19/10/18}

\subsection{Regular surfaces}

\begin{definition}
A \textbf{regular surface} $ S \subset \RR^3 $ is a subset of $ \RR^3 $ such that for all $ p \in S $ there exists
\begin{itemize}
\item an open neighbourhood $ V \subset \RR^3 $ of $ p $,
\item an open set $ U \subset \RR^2 $, and
\item a smooth map $ \phi : U \to \RR^3 $ such that
\begin{itemize}
\item $ \phi\br{U} = V \cap S $,
\item $ \phi $ is a homeomorphism onto its image, and
\item for all $ q \in U $, $ \d\phi_q : \RR^2 \to \RR^3 $ is injective.
\end{itemize}
\end{itemize}
$ \br{U, \phi} $ is called a \textbf{chart} near $ p $ or at $ p $.
\end{definition}

Write $ \phi\br{u, v} = \br{x\br{u, v}, y\br{u, v}, z\br{u, v}} $, then $ \d\phi_q : \RR^2 \to \RR^3 $ has a matrix
$$ \eval{\threebyone{\tpd{x}{u} & \tpd{x}{v}}{\tpd{y}{u} & \tpd{y}{v}}{\tpd{z}{u} & \tpd{z}{v}}}_q, $$
and is injective if and only if the matrix has rank two, if and only if the columns are linearly independent. Then $ \br{1, 0} $ at $ q \in U \subset \RR^2 $ maps to $ \d\phi_q\br{1, 0} $ at $ \phi\br{q} \in S \subset \RR^3 $. Each column matrix of the matrix $ \d\phi_q $ is a tangent vector to $ S $ at $ \phi\br{q} $, so the regularity condition for surfaces ensures that a regular surface has a tangent plane at every point, since the columns of $ \d\phi_q $ span a two-dimensional space, that is a plane in $ \RR^3 $.

\begin{example*}
\hfill
\begin{itemize}
\item Let $ \RR^2 \subset \RR^3 $. Take the chart $ \br{U, \phi} $ with $ U = \RR^2 $ and $ \phi\br{u, v} = \br{u, v, 0} $.
\item Let $ f : \RR^2 \to \RR $ be a smooth function. The \textbf{graph} of $ f $ is
$$ \Gamma_f = \cbr{\br{x, y, f\br{x, y}} \st x, y \in \RR^2}. $$
Take a chart $ \br{U, \phi} $ where $ U = \RR^2 $ and $ \phi\br{u, v} = \br{u, v, f\br{u, v}} $. This is smooth, a homeomorphism onto its image, by projecting to $ \br{x, y} $ or $ \br{u, v} $ plane, and regular, since
$$ \d\phi_{\br{u, v}} = \threebyone{1 & 0}{0 & 1}{\tpd{f}{u} & \tpd{f}{v}} $$
has rank two.
\item $ S = \cbr{z^2 = x^2 + y^2 \st z \ge 0} $ has no smooth chart near $ \br{0, 0, 0} $. Suppose there exists a smooth chart $ \br{U, \phi} $ near $ \br{0, 0, 0} $. Consider the curve $ z = \abs{y} $ and $ x = 0 $. It has a discontinuous velocity at $ t = 0 $. Any smooth curve through $ q \in U $, when $ \phi\br{q} = 0 $, maps to a smooth curve in $ \RR^2 $, by assumption, since we assumed $ \phi $ is smooth, a contradiction. Thus curves like this are not smooth.
\item Define
$$ \function[\phi]{\br{0, 2\pi} \times \br{-1, 1}}{\RR^3}{\br{u, v}}{\br{\sin u, \sin 2u, v}}. $$
Then $ \phi $ is not a homeomorphism onto its image.
\end{itemize}
\end{example*}

\lecture{8}{Monday}{22/10/18}

The following is a very common situation.

\begin{definition}
Let $ F : \RR^3 \to \RR $ be a smooth function. We say that $ S \subset \RR^3 $ is a \textbf{regular level set} of $ F $ if $ S = F^{-1}\br{c} $ for some $ c \in \RR $ and $ \nabla F\br{p} \ne 0 $ for all $ p \in S $.
\end{definition}

\begin{example*}
Let
$$ S = \cbr{x^2 + y^2 + z^2 = 1} \subset \RR^3. $$
Here $ F = x^2 + y^2 + z^2 $ for $ c = 1 $ and $ \nabla F = \br{2x, 2y, 2z} $. Note $ \nabla F\br{p} \ne 0 $ for all $ p \in S $.
\end{example*}

\pagebreak

\begin{proposition}
\label{prop:regularlevelsets}
Regular level sets are regular surfaces.
\end{proposition}

For this we need the inverse function theorem.

\begin{theorem}[Inverse function theorem]
Let $ f : \RR^n \to \RR^n $ be smooth on an open neighbourhood $ V $ of $ p \in \RR^n $, and suppose that $ \d f_p : \RR^n \to \RR^n $ is invertible. Then there exists an open neighbourhood $ U \subset V $ of $ p $ such that $ \eval{f}_U : U \to f\br{U} $ is a \textbf{diffeomorphism}, which is a smooth bijection with a smooth inverse.
\end{theorem}

\begin{proof}[Proof of Proposition \ref{prop:regularlevelsets}]
The goal is to construct a chart on $ S $ near each $ p \in S $. Recall $ S = F^{-1}\br{c} $. Now $ \nabla F\br{p} \ne 0 $, so at least one of $ \tpd{F}{x}\br{p}, \tpd{F}{y}\br{p}, \tpd{F}{z}\br{p} $ is non-zero. Without loss of generality assume $ \tpd{F}{z}\br{p} \ne 0 $. Consider
$$ \function[g]{\RR^3}{\RR^3}{\br{x, y, z}}{\br{x, y, F\br{x, y, z}}}. $$
Then $ g $ is smooth, and
$$ \d g_p = \threebythree{1}{0}{0}{0}{1}{0}{\tpd{F}{x}\br{p}}{\tpd{F}{y}\br{p}}{\tpd{F}{z}\br{p}} $$
is invertible. On a neighbourhood $ U $ of $ p $, $ g $ is a diffeomorphism. Let
$$ W = \cbr{\br{u, v} \in \RR^2 \st \br{u, v, c} \in g\br{U}}. $$
Identify this with $ \cbr{z = c} \cap g\br{U} $. Let $ V = U \cap F^{-1}\br{c} $ and let
$$ \function[\phi]{W}{V}{\br{u, v}}{g^{-1}\br{u, v, c}}. $$
Then $ \phi $ is a chart on $ S $ near $ p $, since $ \phi $ is smooth, a homeomorphism onto its image, and $ \d\phi_q $ is injective for all $ q \in W $, guaranteed by the inverse function theorem.
\end{proof}

\begin{example*}
The torus is a regular surface. Consider the surface obtained by rotating the circle of radius one in the $ xz $ plane, centred at $ \br{2, 0} $, about $ z $-axis. Writing this in cylindrical polars,
$$ S = \cbr{\br{r - 2}^2 + z^2 = 1} = F^{-1}\br{1}, \qquad F\br{x, y, z} = \br{\sqrt{x^2 + y^2} - 2}^2 + z^2. $$
Now,
$$ \nabla F = \br{\tfrac{2x\br{\sqrt{x^2 + y^2} - 2}}{\sqrt{x^2 + y^2}}, \tfrac{2y\br{\sqrt{x^2 + y^2} - 2}}{\sqrt{x^2 + y^2}}, 2z}. $$
If $ \nabla F = 0 $ then $ 2z = 0 $ and either $ x = y = 0 $, so $ r = 0 $, or $ \sqrt{x^2 + y^2} = 2 $, so $ r = 2 $. So $ \nabla F\br{p} \ne 0 $ for all $ p \in S $. Thus $ S $ is a regular surface.
\end{example*}

\begin{proposition}
If $ S $ is a regular surface then, for every $ p \in S $, there exists a neighbourhood $ V $ of $ p $ in $ S $ such that $ V $ is the graph of a smooth function $ z = f\br{x, y} $ or $ y = g\br{x, z} $ or $ x = h\br{y, z} $.
\end{proposition}

\begin{proof}
Take a chart on $ S $ at $ p $. Write
$$ \function[\phi]{U \subset \RR^2}{S \subset \RR^3}{\br{u, v}}{\br{x\br{u, v}, y\br{u, v}, z\br{u, v}}}. $$
Let $ q = \phi^{-1}\br{p} $. Now $ \d\phi_q $ has rank two. So one of the $ 2 \times 2 $ minors of $ \d\phi_q $ is non-zero. Without loss of generality assume
$$ \abs{\twobytwo{\tpd{x}{u}\br{q}}{\tpd{x}{v}\br{q}}{\tpd{y}{u}\br{q}}{\tpd{y}{v}\br{q}}} \ne 0. $$
Consider
$$
\begin{array}{rcccl}
\RR^2 & \xleftarrow{g} & U & \xrightarrow{\phi} & \RR^3 \\
\br{x\br{u, v}, y\br{u, v}} & \mapsfrom & \br{u, v} & \mapsto & \br{x, y, F\br{x, y}}
\end{array},
$$
Then $ g $ is a diffeomorphism near $ q $, by the inverse function theorem, so
$$ \br{\phi \circ g^{-1}}\br{x, y} = \br{x, y, F\br{x, y}}, \qquad F\br{x, y} = z\br{u\br{x, y}, v\br{x, y}}. $$
Thus, near $ p $, $ S $ is the graph $ z = F\br{x, y} $.
\end{proof}

\pagebreak

\subsection{Tangent vectors and tangent planes}

\lecture{9}{Tuesday}{23/10/18}

\begin{proposition}
Let $ S $ be a regular surface, let $ p \in S $, and let $ \alpha : \br{-\epsilon, \epsilon} \to \RR^3 $ be a regular curve in $ S $ with $ \alpha\br{0} = p $. Let $ \phi : U \to \RR^3 $ be a chart on $ S $ near $ p $. There exist smooth functions $ u, v : \br{-\epsilon', \epsilon'} \to U $ such that $ \alpha\br{t} = \phi\br{u\br{t}, v\br{t}} $ for $ t \in \br{-\epsilon', \epsilon'} $, so
$$
\begin{tikzcd}
& \br{-\epsilon, \epsilon} \arrow{dl}[swap]{\br{u\br{t}, v\br{t}}} \arrow{dr}{\alpha} & \\
q \in U \arrow{rr}[swap]{\phi} & & S \ni p
\end{tikzcd}.
$$
\end{proposition}

\begin{proof}
Without loss of generality $ S $ has the form $ z = f\br{x, y} $. Then
$$ \phi\br{u, v} = \br{x\br{u, v}, y\br{u, v}, f\br{x\br{u, v}, y\br{u, v}}}. $$
Let $ q = \phi^{-1}\br{p} $, and set $ q = \br{u_0, v_0} $. Define
$$ \function[F]{U \times \br{-\epsilon, \epsilon}}{\RR^3}{\br{u, v, t}}{\br{x\br{u, v}, y\br{u, v}, f\br{x\br{u, v}, y\br{u, v}} + t}}. $$
Then $ F\br{u_0, v_0, 0} = p $, and
$$ \d F_{\br{u_0, v_0, 0}} = \threebythree{\tpd{x}{u}}{\tpd{x}{v}}{0}{\tpd{y}{u}}{\tpd{y}{v}}{0}{\tpd{f}{x}\tpd{x}{u} + \tpd{f}{y}\tpd{y}{u}}{\tpd{f}{x}\tpd{x}{v} + \tpd{f}{y}\tpd{y}{v}}{1}. $$
The first two columns of $ \d\phi_{\br{u_0, v_0}} $ has rank two, so $ \br{\tpd{x}{u}, \tpd{y}{u}} $ and $ \br{\tpd{x}{v}, \tpd{y}{v}} $ are linearly independent, so $ \d F_{\br{u_0, v_0, 0}} $ is invertible. Applying the inverse function theorem, $ F $ is a diffeomorphism in a neighbourhood of $ \br{u_0, v_0, 0} $. Let $ G = F^{-1} $ near $ p $. Then
$$ G\br{x\br{u, v}, y\br{u, v}, f\br{x\br{u, v}, y\br{u, v}}} = \br{u, v, 0}. $$
Now $ \alpha\br{t} = \br{x\br{t}, y\br{t}, f\br{x\br{t}, y\br{t}}} $ for some smooth functions $ x $ and $ y $. Setting $ \br{u\br{t}, v\br{t}} = \br{G \circ \alpha}\br{t} $ we find $ u $ and $ v $ are smooth, because $ G $ and $ \alpha $ are, and $ \alpha\br{t} = \phi\br{u\br{t}, v\br{t}} $.
\end{proof}

\begin{definition}
Let $ S $ be a regular surface, and let $ p \in S $. A \textbf{tangent vector} to $ S $ at $ p $ is a vector in $ \RR^3 $ of the form $ \alpha'\br{0} $, where $ \alpha : \br{-\epsilon, \epsilon} \to S $ is a regular curve with $ \alpha\br{0} = p $. The \textbf{tangent plane} $ \T_p S $ to $ S $ at $ p $ is the set of all tangent vectors.
\end{definition}

This is independent of charts. How to compute this?

\begin{proposition}
If $ \phi : U \to \RR^3 $ is a chart for $ S $ at $ p $, with $ \phi\br{q} = p $, then $ \T_p S $ is spanned by the columns of $ \d\phi_q $.
\end{proposition}

\begin{proof}
Write $ \d\phi_q = \br{w_1, w_2} $. For all $ a, b \in \RR $, need to show $ aw_1 + bw_2 \in \T_p S $. Consider $ \alpha\br{t} = \phi\br{q + t\br{a, b}} $. Chain rule implies that $ \alpha'\br{0} = \d\phi_q \cdot \br{a, b} = aw_1 + bw_2 $. So $ aw_1 + bw_2 \in \T_p S $. Conversely, need to show any element of $ \T_p S $ is of the form $ aw_1 + bw_2 $ for some $ a, b \in \RR $. This is of the form $ \alpha'\br{0} $ where $ \alpha : \br{-\epsilon, \epsilon} \to S $ is a regular curve with $ \alpha\br{0} = p $. Given the discussion earlier, we know that there exist smooth functions $ u, v : \br{-\epsilon', \epsilon'} \to U $ such that $ \alpha\br{t} = \phi\br{u\br{t}, v\br{t}} $ for $ t \in \br{-\epsilon', \epsilon} $. Then
$$ \alpha'\br{0} = \d\phi_{\br{u\br{0}, v\br{0}}} \cdot \twobyone{\tod{u}{t}\br{0}}{\tod{v}{t}\br{0}} = \d\phi_q \cdot \twobyone{a}{b}, \qquad a = \dod{u}{t}\br{0}, \qquad b = \dod{v}{t}\br{0}. $$
Thus $ \alpha'\br{0} = aw_1 + bw_2 $ as claimed.
\end{proof}

\begin{exercise*}
Consider
$$ S = \cbr{\br{x, y, z} \st x^2 + y^2 + z^2 = 1}, \qquad p = \br{0, 0, 1}. $$
Use the chart
$$ \function[\phi]{U}{\RR^3}{\br{u, v}}{\br{u, v, \sqrt{1 - u^2 - v^2}}}, $$
where $ U $ is the open unit disc in $ \RR^2 $ to compute $ \T_p S $ as the $ xy $ plane.
\end{exercise*}

\pagebreak

\lecture{10}{Friday}{26/10/18}

\begin{proposition}
Let $ S = F^{-1}\br{c} $ for $ F : \RR^3 \to \RR $ then for any $ p \in S $,
$$ \T_p S = \br{\nabla F\br{p}}^\perp. $$
\end{proposition}

\begin{proof}
Let $ \alpha : \br{-\epsilon, \epsilon} \to S $ such that $ \alpha\br{0} = p $, so $ \alpha'\br{0} \in \T_p S $ and $ F\br{\alpha\br{t}} = c $. Let $ \alpha = \br{x\br{t}, y\br{t}, z\br{t}} $. Then
$$ 0 = \eval{\dod{}{t}F\br{\alpha\br{t}}}_{t = 0} = \eval{\br{\dpd{F}{x}\dod{x}{t} + \dpd{F}{y}\dod{y}{t} + \dpd{F}{z}\dod{z}{t}}}_{t = 0} = \nabla F\br{\alpha\br{0}} \cdot \alpha'\br{0}, $$
so $ \alpha'\br{0} \in \br{\nabla F\br{p}}^\perp $, so $ \T_p S \subseteq \br{\nabla F\br{p}}^\perp $, which is two-dimensional. This implies that $ \T_p S = \br{\nabla F\br{p}}^\perp $.
\end{proof}

\begin{example*}
Let $ S $ be the paraboloid
$$ \cbr{z = x^2 + y^2} = F^{-1}\br{0}, \qquad F = x^2 + y^2 - z, $$
and let $ p = \br{1, 3, 10} $. Then $ \nabla F = \br{2x, 2y, -1} \ne 0 $, and $ \nabla F\br{p} = \br{2, 6, -1} $, so
$$ \T_p S = \br{\nabla F\br{p}}^\perp = \cbr{2x + 6y - z = 0}. $$
\end{example*}

\begin{definition}
Let $ S_1 $ and $ S_2 $ be regular surfaces. A map $ F : S_1 \to \RR^3 $ is \textbf{smooth} if for every chart $ \phi : U \to S_1 $ the composition $ U \xrightarrow{\phi} S_1 \xrightarrow{F} \RR^3 $ is smooth. A map $ F : S_1 \to S_2 $ is \textbf{smooth} if it is smooth when viewed as a map $ F : S_1 \to \RR^3 $. The \textbf{differential} of a smooth map $ F : S_1 \to S_2 $ at $ p \in S_1 $, denoted
$$ \d F_p : \T_p S_1 \to \T_{F\br{p}} S_2, $$
is defined as follows. Let $ v = \alpha'\br{0} $ for $ \alpha : \br{-\epsilon, \epsilon} \to S_1 $ such that $ \alpha\br{0} = p $ and $ \beta\br{t} = F\br{\alpha\br{t}} : \br{-\epsilon, \epsilon} \to S_2 $. This is a regular curve in $ S_2 $, so $ \beta'\br{0} \in \T_{F\br{p}} S_2 $, since $ \beta\br{0} = F\br{p} $. We define
$$ \d F_p\br{v} = \beta'\br{0}. $$
\end{definition}

\begin{proposition}
This definition is independent of the choice of curve $ \alpha $.
\end{proposition}

\begin{proof}
Let $ \alpha_1, \alpha_2 : \br{-\epsilon, \epsilon} \to S_1 $ such that $ \alpha_1\br{0} = p $ and $ \alpha_1'\br{0} = \alpha_2'\br{0} $. Let $ \alpha_1\br{t} = \phi\br{u_1\br{t}, v_1\br{t}} $ and $ \alpha_2\br{t} = \phi\br{u_2\br{t}, v_2\br{t}} $, so
$$ \dpd{\phi}{u_1}\dod{u_1}{t}\br{0} + \dpd{\phi}{v_1}\dod{v_1}{t}\br{0} = \alpha_1'\br{0} = \alpha_2'\br{0} = \dpd{\phi}{u_2}\dod{u_2}{t}\br{0} + \dpd{\phi}{v_2}\dod{v_2}{t}\br{0}. $$
Recall $ \tod{\phi}{u} $ and $ \tod{\phi}{v} $ are linearly independent. This implies that
\begin{equation}
\label{eq:1}
\dod{u_1}{t}\br{0} = \dod{u_2}{t}\br{0}, \qquad \dod{v_1}{t}\br{0} = \dod{v_2}{t}\br{0}.
\end{equation}
Want $ \d F_p\br{\alpha_1'\br{0}} = \d F_p\br{\alpha_2'\br{0}} $.
$$ \d F_p\br{\alpha_i\br{0}} = \eval{\dod{}{t}F\br{\phi\br{u_i, v_i}}}_{t = 0} = \eval{\br{\dpd{\br{F \circ \phi}}{u}\dod{u_i}{t} + \dpd{\br{F \circ \phi}}{v}\dod{v_i}{t}}}_{t = 0}, \qquad i = 1, 2, $$
so $ \d F_p\br{\alpha_1'\br{0}} = \d F_p\br{\alpha_2'\br{0}} $ by $ \br{\ref{eq:1}} $. It follows that $ \d F_p\br{v} $ is the same if we define it via $ \alpha_1 $ or $ \alpha_2 $.
\end{proof}

\begin{proposition}
$ \d F_p : \T_p S_1 \to \T_{F\br{p}} S_2 $ is linear.
\end{proposition}

\begin{proof}
Let $ v, w \in \T_p S_1 $ and $ c, d \in \RR $. Want $ \d F_p\br{cv + dw} = c\d F_p\br{v} + d\d F_p\br{w} $. Let $ \alpha_1, \alpha_2 : \br{-\epsilon, \epsilon} \to S_1 $ such that $ v = \alpha_1'\br{0} $ and $ w = \alpha_2'\br{0} $. Let $ \phi : U \to S_1 $ be a chart. Without loss of generality we assume $ \br{0, 0} \in U $ and $ \phi\br{0, 0} = p $. Let $ \alpha_i\br{t} = \phi\br{u_i\br{t}, v_i\br{t}} $. Let
$$ \alpha_3\br{t} = \phi\br{cu_1\br{t} + du_2\br{t}, cv_1\br{t} + dv_2\br{t}}, $$
so $ \alpha_3\br{0} = \phi\br{0, 0} = p $ and
\begin{align*}
\alpha_3'\br{0}
& = \dpd{\phi}{u}\br{cu_1'\br{0} + du_2'\br{0}} + \dpd{\phi}{v}\br{cv_1'\br{0} + dv_2'\br{0}} \\
& = c\br{u_1'\br{0}\dpd{\phi}{u} + v_1'\br{0}\dpd{\phi}{v}} + d\br{u_2'\br{0}\dpd{\phi}{u} + v_2'\br{0}\dpd{\phi}{v}}
= c\alpha_1'\br{0} + d\alpha_2'\br{0}
= cv + dw.
\end{align*}

\pagebreak

Then
\begin{align*}
\d F_p\br{cv + dw}
& = \d F_p\br{\alpha_3'\br{0}}
= \br{\br{F \circ \phi}\br{cu_1\br{t} + du_2\br{t}, cv_1\br{t} + dv_2\br{t}}}'\br{0} \\
& = \dpd{\br{F \circ \phi}}{u}\br{cu_1'\br{0} + du_2'\br{0}} + \dpd{\br{F \circ \phi}}{u}\br{cv_1'\br{0} + dv_2'\br{0}} \\
& = c\br{\dpd{\br{F \circ \phi}}{u}u_1'\br{0} + \dpd{\br{F \circ \phi}}{v}v_1'\br{0}} + d\br{\dpd{\br{F \circ \phi}}{u}u_2'\br{0} + \dpd{\br{F \circ \phi}}{v}v_2'\br{0}} \\
& = c\br{F \circ \alpha_1}'\br{0} + d\br{F \circ \alpha_2}'\br{0}
= c\d F_p\br{v} + d\d F_p\br{w},
\end{align*}
since
$$ \d F_p\br{\alpha_i'\br{0}} = \br{F \circ \alpha_i}'\br{0} = \br{\dpd{\br{F \circ \phi}}{u}}\dod{u_i}{t} + \br{\dpd{\br{F \circ \phi}}{v}}\dod{v_i}{t}. $$
\end{proof}

\begin{remark*}
We will use the following identity. Given a chart $ \phi : U \to S_1 $ and $ F : S_1 \to S_2 $,
$$ \d F_p\br{\dpd{\phi}{u}\br{u_0, v_0}} = \dpd{\br{F \circ \phi}}{u}\br{u_0, v_0}, \qquad \d F_p\br{\dpd{\phi}{v}\br{u_0, v_0}} = \dpd{\br{F \circ \phi}}{v}\br{u_0, v_0}. $$
\end{remark*}

Similarly, we define the \textbf{differential} of a smooth function $ f : S \to \RR $ if $ v \in \T_p S $ satisfies $ v = \alpha'\br{0} $ for $ \alpha : \br{-\epsilon, \epsilon} \to S $ as
$$ \d f_p\br{v} = f\br{\alpha\br{t}}'\br{0}. $$

\begin{exercise*}
Show that $ \d f_p $ is well-defined, so independent of choice of $ \alpha $, and that it is linear.
\end{exercise*}

\lecture{11}{Monday}{29/10/18}

\begin{example*}
Let
$$ S = \cbr{x^2 + y^2 + z^2 = 1} = F^{-1}\br{1}, \qquad \function[F]{\RR^3}{\RR}{\br{x, y, z}}{x^2 + y^2 + z^2}, $$
and let
$$ p = \br{0, 1, 0}, \qquad \function[f]{S}{\RR}{\br{x, y, z}}{z}. $$
Then $ \nabla F\br{p} = \br{0, 2, 0} $, so
$$ \T_p S = \br{\nabla F\br{p}}^\perp = \abr{\br{1, 0, 0}, \br{0, 0, 1}}. $$
But to compute $ \d f_p $ use the chart near $ p $,
$$ \function[\phi_p]{U}{S}{\br{u, v}}{\br{u, \sqrt{1 - u^2 - v^2}, v}}, \qquad U = \cbr{u^2 + v^2 < 1}. $$
Then $ \br{f \circ \phi}\br{u, v} = v $, so
$$ \d f_p\br{\dpd{\phi}{u}\br{0, 0}} = 0, \qquad \d f_p\br{\dpd{\phi}{v}\br{0, 0}} = 1. $$
This determines $ \d f_p : \T_p S \to \RR $, a linear map.
\end{example*}

\begin{proposition}
Let $ f : S_1 \to S_2 $ be a smooth map. If $ \d f_p : \T_p S_1 \to \T_{f\br{p}} S_2 $ is an isomorphism, then there exists an open neighbourhood $ V \subseteq S_1 $ of $ p $ such that $ f : V \to f\br{V} $ is a diffeomorphism.
\end{proposition}

\begin{proof}
Take charts $ \phi_1 : U_1 \to S_1 $ at $ p $ and $ \phi_2 : U_2 \to S_2 $ at $ f\br{p} $, so
$$
\begin{tikzcd}
\phi_1\br{q_1} = p \in \phi_1\br{U_1} \subseteq S_1 \arrow{r}{f} & \phi_2\br{q_2} = f\br{p} \in \phi_2\br{U_2} \subseteq S_2 \\
q_1 \in U_1 \subseteq \RR^2 \arrow{u}{\phi_1} \arrow{r}[swap]{g} & q_2 \in U_2 \subseteq \RR^2 \arrow{u}[swap]{\phi_2}
\end{tikzcd}.
$$
Check that the map
$$ g : U_1 \to S_1 \to S_2 \to U_2 $$
satisfies $ g\br{q_1} = q_2 $, and $ \d g_{q_1} $ is invertible. Use the inverse function theorem for $ g $. Then $ f = \phi_2 \circ g \circ \phi_1^{-1} $ will be a diffeomorphism on some neighbourhood of $ \phi_1\br{q_1} = p $.
\end{proof}

\pagebreak

\subsection{Normal vectors}

A regular surface $ S \subseteq \RR^3 $ has two normal vectors at $ p $ in $ S $. If $ S = F^{-1}\br{c} $ for some smooth function $ F : \RR^3 \to \RR $, then $ \T_p S = \br{\nabla F\br{p}}^\perp $. We can define the \textbf{unit normal vector}
$$ \N\br{p} = \dfrac{\nabla F\br{p}}{\abs{\nabla F\br{p}}}. $$
In general we can only do this locally, in a chart. Given a chart $ \phi : U \to S $ such that $ \phi\br{q} = p $, we know that $ \T_p S $ is spanned by $ \tod{\phi}{u}\br{q} $ and $ \tod{\phi}{v}\br{q} $. Define
$$ \N\br{p} = \dfrac{\tpd{\phi}{u}\br{q} \times \tpd{\phi}{v}\br{q}}{\abs{\tpd{\phi}{u}\br{q} \times \tpd{\phi}{v}\br{q}}}. $$
This can always be done locally, but not globally. On the M\"obius strip, you cannot pick $ \N\br{p} $ continuously.

\begin{definition}
$ S $ is \textbf{orientable} if it admits a continuous choice of unit normal vector $ \N\br{p} \in S $. If $ S $ is orientable, we get a map $ \N : S \to \S^2 $. Then $ \N $ is called the \textbf{Gauss map}. If this exists, it is smooth.
\end{definition}

\begin{example*}
Let $ S = \S^2 $ be the unit sphere. As $ S = F^{-1}\br{1} $ for $ F\br{x, y, z} = x^2 + y^2 + z^2 $,
$$ \N\br{x, y, z} = \dfrac{\nabla F\br{x, y, z}}{\abs{\nabla F\br{x, y, z}}} = \dfrac{\br{2x, 2y, 2z}}{\sqrt{4x^2 + 4y^2 + 4z^2}} = \br{x, y, z}. $$
Or, near the north pole $ \br{0, 0, 1} $ we have a chart $ \phi\br{u, v} = \br{u, v, \sqrt{1 - u^2 - v^2}} $, so
$$ \dpd{\phi}{u} = \br{1, 0, \tfrac{-u}{\sqrt{1 - u^2 - v^2}}}, \qquad \dpd{\phi}{u} = \br{0, 1, \tfrac{-v}{\sqrt{1 - u^2 - v^2}}}, $$
so
$$ \dpd{\phi}{u} \times \dpd{\phi}{v} = \br{\tfrac{u}{\sqrt{1 - u^2 - v^2}}, \tfrac{v}{\sqrt{1 - u^2 - v^2}}, 1}. $$
Thus
$$ \N\br{\phi\br{u, v}} = \dfrac{\tpd{\phi}{u} \times \tpd{\phi}{v}}{\abs{\tpd{\phi}{u} \times \tpd{\phi}{v}}} = \br{u, v, \sqrt{1 - u^2 - v^2}} = \phi\br{u, v}. $$
\end{example*}

\begin{example*}
Let
$$ S = \cbr{ax + by + cz = d} = F^{-1}\br{d}, \qquad F\br{x, y, z} = ax + by + cz $$
be a plane. Then
$$ \N\br{x, y, z} = \dfrac{\nabla F\br{x, y, z}}{\abs{\nabla F\br{x, y, z}}} = \dfrac{\br{a, b, c}}{\sqrt{a^2 + b^2 + c^2}} $$
is a constant map.
\end{example*}

What is $ \d\N_p : \T_p S \to \T_{\N\br{p}} \S^2 $? $ \T_{\N\br{p}} \S^2 $ is all vectors orthogonal to $ \N\br{p} $, which is $ \T_p S $. We use this identification to write $ \d\N_p : \T_p S \to \T_p S $.

\begin{example*}
Let
$$ S = \cbr{\br{x, y, z} \st x^2 + y^2 + z^2 = r^2} = F^{-1}\br{r^2}, \qquad F = x^2 + y^2 + z^2 $$
be a sphere of radius $ r $, and let $ p = \br{x, y, z} $. Then
$$ \N\br{p} = \dfrac{\nabla F\br{p}}{\abs{\nabla F\br{p}}} = \br{\dfrac{x}{r}, \dfrac{y}{r}, \dfrac{z}{r}} = \dfrac{p}{r}. $$
Let $ \alpha : \br{-\epsilon, \epsilon} \to S $ be a curve with $ \alpha\br{0} = p $, so
$$ \d\N_p\br{\alpha'\br{0}} = \eval{\dod{\br{\br{\N \circ \alpha}\br{t}}}{t}}_{t = 0} = \dfrac{1}{r}\alpha'\br{0}. $$
Thus $ \d\N_p : \T_p S \to \T_p S $ is $ \d\N_p = \id / r $.
\end{example*}

\pagebreak

\section{Curvature}

\lecture{12}{Tuesday}{30/10/18}

We will try to define the curvature using the Gauss map.

\subsection{The second fundamental form}

\begin{example*}
Let
$$ S = \cbr{ax + by + cz = d} $$
be a plane, and let $ p \in S $. Then
$$ \N\br{p} = \dfrac{\br{a, b, c}}{\sqrt{a^2 + b^2 + c^2}} $$
is constant with respect to $ p $, so $ \d\N_p : \T_p S \to \T_{\N\br{p}} \S^2 $ is the zero map.
\end{example*}

\begin{exercise*}
Let
$$ S = \cbr{x^2 + 2y^2 + z^2 = r^2}. $$
Compute $ \d\N_p $ at $ \br{1, 0, 0} $ and at $ \br{0, 1, 0} $.
\end{exercise*}

\begin{definition}
Let $ S $ be a regular orientable surface, let $ p \in S $. The \textbf{second fundamental form} at $ p $ is
$$ \function[\A]{\T_p S \times \T_p S}{\RR}{\br{X, Y}}{-X \cdot \d\N_p\br{Y}}, $$
which is how much $ \N_p\br{Y} $ is changing in the direction of $ X $.
\end{definition}

\begin{example*}
If $ S $ is a sphere of radius $ r $ then
$$ \A\br{X, Y} = -\dfrac{1}{r}X \cdot Y. $$
\end{example*}

\begin{proposition}
$ \A $ is a symmetric bilinear form.
\end{proposition}

\begin{proof}
\hfill
\begin{itemize}
\item Bilinearity.
\begin{align*}
\A\br{X, cY_1 + dY_2}
& = -X \cdot \d\N_p\br{cY_1 + dY_2}
= -X \cdot \br{c\d\N_p\br{Y_1} + d\d\N_p\br{Y_2}} \\
& = c\br{-X \cdot \d\N_p\br{Y_1}} + d\br{-X \cdot \d\N_p\br{Y_2}}
= c\A\br{X, Y_1} + d\A\br{X, Y_2}.
\end{align*}
Similarly $ \A\br{cX_1 + dX_2, Y} = c\A\br{X_1, Y} + d\A\br{X_2, Y} $.
\item Symmetry. Take a chart $ \phi : U \to S $ near $ p $. Then $ \tod{\phi}{u}\br{q} $ and $ \tod{\phi}{v}\br{q} $, where $ \phi\br{q} = p $, form a basis for the tangent space $ \T_p S $. It is sufficient to prove that
$$ \A\br{\dpd{\phi}{u}\br{q}, \dpd{\phi}{v}\br{q}} = \A\br{\dpd{\phi}{v}\br{q}, \dpd{\phi}{u}\br{q}}. $$
Observe that
$$ \N\br{\phi\br{p}} \cdot \dpd{\phi}{u} = 0, \qquad \N\br{\phi\br{p}} \cdot \dpd{\phi}{v} = 0, $$
since $ \tod{\phi}{u} $ and $ \tod{\phi}{v} $ are in $ \T_p S $ and $ \N\br{\phi\br{p}} $ is normal to $ \T_p S $. Applying $ \tpd{}{v} $ and $ \tpd{}{u} $,
$$ \dpd{}{v}\br{\N \circ \phi} \cdot \dpd{\phi}{u} + \br{\N \circ \phi} \cdot \dmd{\phi}{2}{v}{}{u}{} = 0, \qquad \dpd{}{u}\br{\N \circ \phi} \cdot \dpd{\phi}{v} + \br{\N \circ \phi} \cdot \dmd{\phi}{2}{u}{}{v}{} = 0. $$
Then
$$ \A\br{\dpd{\phi}{u}, \dpd{\phi}{v}} = \br{\N \circ \phi} \cdot \dmd{\phi}{2}{v}{}{u}{}, \qquad \A\br{\dpd{\phi}{v}, \dpd{\phi}{u}} = \br{\N \circ \phi} \cdot \dmd{\phi}{2}{u}{}{v}{}. $$
Thus both these are equal, so
$$ \A\br{\dpd{\phi}{u}, \dpd{\phi}{v}} = \A\br{\dpd{\phi}{v}, \dpd{\phi}{u}}. $$
\end{itemize}
\end{proof}

\lecture{13}{Friday}{02/11/18}

Lecture 13 is a problems class.

\pagebreak

\subsection{Normal curvature, Gaussian curvature, and mean curvature}

\lecture{14}{Monday}{05/11/18}

There exists an orthonormal basis $ x_1 $ and $ x_2 $ for $ \T_p S $ such that
$$ \d\N_p\br{x_i} = -\lambda_ix_i, \qquad i = 1, 2. $$
Then
\begin{itemize}
\item $ \A\br{x_1, x_1} = -x_1 \cdot \d\N_p\br{x_1} = x_1 \cdot \br{\lambda_1x_1} = \lambda_1 $,
\item $ \A\br{x_1, x_2} = -x_1 \cdot \d\N_p\br{x_2} = x_1 \cdot \br{\lambda_2x_2} = 0 $, and
\item $ \A\br{x_2, x_2} = -x_2 \cdot \d\N_p\br{x_2} = x_2 \cdot \br{\lambda_2x_2} = \lambda_2 $.
\end{itemize}
We call $ x_1 $ and $ x_2 $ the \textbf{principal directions} in $ \T_p S $ and $ \lambda_1 $ and $ \lambda_2 $ the \textbf{principal curvatures}.

\begin{lemma}
Let $ S $ be a regular surface and $ \lambda_1\br{p} \le \lambda_2\br{p} $ be the principal curvatures at $ p \in S $. Then
$$ \lambda_1 = \min\cbr{\A\br{x, x} \st x \in \T_p S, \ \abs{x} = 1}, \qquad \lambda_2 = \max\cbr{\A\br{x, x} \st x \in \T_p S, \ \abs{x} = 1}. $$
\end{lemma}

\begin{proof}
Consider $ x \in \T_p S $ with $ \abs{x} = 1 $. Write $ x = c_1x_1 + c_2x_2 $ where $ c \in \RR $, then $ c_1^2 + c_2^2 = 1 $. Then
\begin{align*}
\A\br{x, x}
& = \A\br{c_1x_1 + c_2x_2, c_1x_1 + c_2x_2} \\
& = c_1^2\A\br{x_1, x_1} + c_1c_2\A\br{x_1, x_2} + c_2c_1\A\br{x_2, x_1} + c_2^2\A\br{x_2, x_2} \\
& = \lambda_1c_1^2 + \lambda_2c_2^2
\le \lambda_2c_1^2 + \lambda_2c_2^2
= \lambda_2,
\end{align*}
with equality if and only if $ c_1 = 0 $ and $ c_2 = 1 $. Similarly
$$ \A\br{x, x} = \lambda_1c_1^2 + \lambda_2c_2^2 \ge \lambda_1c_1^2 + \lambda_1c_2^2 = \lambda_1, $$
with equality if and only if $ c_1 = 1 $ and $ c_2 = 0 $.
\end{proof}

\begin{example*}
Let $ S $ be a sphere of radius $ r $. Then $ \N\br{p} = p / r $ and $ \d\N_p = \id / r $, so principal curvatures are $ \lambda_1 = \lambda_2 = 1 / r $, which are independent of $ p $.
\end{example*}

\begin{example*}
Suppose $ S $ is a connected regular surface and that $ \lambda_1\br{p} = \lambda_2\br{p} = 0 $ for all $ p \in S $. Then $ \d\N_p = 0 $, so $ \N\br{p} $ is constant for all $ p \in S $. Suppose $ \N\br{p} = \vec{v} $ for all $ p \in S $. Then $ S $ is in the plane perpendicular to $ \vec{v} $ through $ p $. \footnote{Exercise}
\end{example*}

\begin{definition}
Suppose $ C \subset S $ is a curve through $ p \in S $. Suppose $ \N $ is the unit normal vector to $ S $ at $ p $ and $ n $ is the unit normal vector to $ C $ at $ p $. The \textbf{normal curvature} of $ C $ at $ p $ is
$$ \k_n\br{p} = \kappa\cos \theta, $$
where $ \theta $ is the angle between $ \N $ and $ n $ and $ \kappa $ is the curvature of $ C $ at $ p $, so $ \k_n\br{p} $ is the length of the projection of $ \vec{\kappa} $, the vector curvature of $ C $ at $ p $, along $ \N $.
\end{definition}

\begin{proposition}
Let $ S $ be a regular surface and $ C \subset S $ be a curve with unit tangent vector $ v \in \T_p S $. Then
$$ \A\br{v, v} = \k_n\br{p}. $$
\end{proposition}

\begin{proof}
Let $ \alpha : \br{-\epsilon, \epsilon} \to S $ be an arc-length parametrisation of $ C $ near $ p $. Then $ \alpha\br{0} = p $ and $ \alpha'\br{0} = v $. We have $ \abr{\alpha'\br{t}, \N\br{\alpha\br{t}}} = 0 $ for all $ t $. Differentiating,
$$ \abr{\alpha''\br{t}, \N\br{\alpha\br{t}}} + \abr{\alpha'\br{t}, \d\N_{\alpha\br{t}}\br{\alpha'\br{t}}}, $$
for all $ t $. Setting $ t = 0 $,
$$ \abr{\kappa n_p, \N\br{p}} + \abr{v, \d\N_p\br{v}} = 0. $$
Thus
$$ \A\br{v, v} = -\abr{v, \d\N_p\br{v}} = \abr{\kappa n_p, \N\br{p}} = \k_n\br{p}. $$
\end{proof}

\pagebreak

\begin{definition}
A point $ p \in S $ is \textbf{umbilical} if and only if the principal curvatures $ \lambda_1\br{p} $ and $ \lambda_2\br{p} $ are equal.
\end{definition}

\begin{proposition}
If $ S $ is a connected regular surface such that every point in $ S $ is umbilical then $ S $ is contained in a plane or a sphere.
\end{proposition}

\lecture{15}{Tuesday}{06/11/18}

\begin{proof}
By assumption, there exists a smooth function $ \lambda : S \to \RR $ such that $ \lambda_1\br{p} = \lambda_2\br{p} = \lambda\br{p} $ for all $ p \in S $.
\begin{enumerate}[leftmargin=0.5in, label=Step \arabic*.]
\item Prove $ \lambda\br{p} $ is constant. Take a chart $ \phi : U \to S $ near $ p $. Then
$$ \dpd{}{u}\br{\N \circ \phi} = \d\N_{\phi\br{u, v}} \cdot \dpd{\phi}{u} = -\br{\lambda \circ \phi}\dpd{\phi}{u}, \qquad \dpd{}{v}\br{\N \circ \phi} = \d\N_{\phi\br{u, v}} \cdot \dpd{\phi}{v} = -\br{\lambda \circ \phi}\dpd{\phi}{v}, $$
since $ \d\N_{\phi\br{u, v}} $ is some multiple of $ \id $. Now
$$ \dmd{}{2}{u}{}{v}{}\br{\N \circ \phi} = \dmd{}{2}{v}{}{u}{}\br{\N \circ \phi}, $$
so
$$ -\dpd{}{u}\br{\lambda \circ \phi}\dpd{\phi}{v} = -\dpd{}{v}\br{\lambda \circ \phi}\dpd{\phi}{u}. $$
The vectors $ \tod{\phi}{v} $ and $ \tod{\phi}{u} $ are linearly independent, so
$$ -\dpd{}{u}\br{\lambda \circ \phi} = -\dpd{}{v}\br{\lambda \circ \phi} = 0, $$
so $ \lambda \circ \phi $ is locally constant, that is $ \lambda $ is constant on $ \phi\br{U} $. Then $ S $ is connected, and covered by such charts, so $ \lambda $ is constant.
\item If $ \lambda\br{p} \equiv 0 $ for all $ p \in S $ then we already know that $ S $ is contained in a plane. Otherwise $ \lambda\br{p} \equiv \lambda_0 $ with $ \lambda_0 \ne 0 $. Without loss of generality $ \lambda_0 > 0 $. Now
$$ \dpd{}{u}\br{\N \circ \phi} = -\lambda_0\dpd{\phi}{u}, \qquad \dpd{}{v}\br{\N \circ \phi} = -\lambda_0\dpd{\phi}{v}, $$
so
$$ \dpd{}{u}\br{\dfrac{1}{\lambda_0}\br{\N \circ \phi} + \phi} = 0, \qquad \dpd{}{v}\br{\dfrac{1}{\lambda_0}\br{\N \circ \phi} + \phi} = 0, $$
so there exists $ c_0 \in \RR^3 $ such that $ \br{\N \circ \phi} / \lambda_0 + \phi = c_0 $. Then $ \br{\N \circ \phi} / \lambda_0 $ has length $ 1 / \lambda_0 $ and $ c_0 $ is constant, so $ \phi\br{u, v} $ lies in the sphere centred at $ c_0 $ radius $ 1 / \lambda_0 $.
\end{enumerate}
\end{proof}

\lecture{16}{Friday}{09/11/18}

What if $ \lambda_1\br{p} \ne \lambda_2\br{p} $?

\begin{definition}
Let $ p \in S $ be a point on a regular surface. Let $ \lambda_1\br{p} $ and $ \lambda_2\br{p} $ be the principal curvatures. Then
$$ \K\br{p} = \lambda_1\br{p}\lambda_2\br{p} = \det \d\N_p $$
is the \textbf{Gaussian curvature} at $ p $ and
$$ \H\br{p} = \dfrac{1}{2}\br{\lambda_1\br{p} + \lambda_2\br{p}} = -\dfrac{1}{2}\Tr \d\N_p $$
is the \textbf{mean curvature} at $ p $.
\end{definition}

Reversing the orientation of $ S $, that is replacing $ \N $ by $ -\N $, sends $ \H\br{p} $ to $ -\H\br{p} $ and $ \K\br{p} $ to $ \K\br{p} $.

\begin{example*}
If $ S $ is a sphere of radius $ r $ then $ \K = 1 / r^2 $ and $ \H = 1 / r $. A plane has $ \K = \H = 0 $.
\end{example*}

\pagebreak

\subsection{What does Gaussian curvature mean?}

Let $ S $ be a regular surface, and let $ p \in S $. Choose a chart $ \phi : U \to S $ on $ S $ near $ p $. Choose a curve through $ p $, $ \gamma\br{t} = \phi\br{ct, dt} $. Now $ \gamma'\br{t} \cdot \N\br{\gamma\br{t}} = 0 $ for all $ t $. Differentiating,
$$ \gamma''\br{t} \cdot \N\br{\gamma\br{t}} + \gamma'\br{t} \cdot \d\N_{\gamma\br{t}}\br{\gamma'\br{t}} = 0, $$
or in other words
$$ \A\br{\gamma'\br{t}, \gamma'\br{t}} = \gamma''\br{t} \cdot \N\br{\gamma\br{t}}. $$
At $ t = 0 $,
$$ \gamma'\br{t} = c\dpd{\phi}{u} + d\dpd{\phi}{v}, \qquad \gamma''\br{t} = c^2\dpd[2]{\phi}{u} + 2cd\dmd{\phi}{2}{u}{}{v}{} + d^2\dpd[2]{\phi}{v}. $$
Thus
$$ \A\br{c\dpd{\phi}{u} + d\dpd{\phi}{v}, c\dpd{\phi}{u} + d\dpd{\phi}{v}} = \br{c^2\dpd[2]{\phi}{u} + 2cd\dmd{\phi}{2}{u}{}{v}{} + d^2\dpd[2]{\phi}{v}} \cdot \N\br{p}. $$

\begin{proposition}
If $ \K\br{p} > 0 $ then all points of $ S $ near $ p $ lie on the same side of $ \T_p S $. If $ \K\br{p} < 0 $ then there are points of $ S $ near $ p $ on each side of $ \T_p S $.
\end{proposition}

\begin{proof}
Consider the Taylor series
$$ \phi\br{u, v} = p + \br{\dpd{\phi}{u}\br{0, 0}u + \dpd{\phi}{v}\br{0, 0}v} + \dfrac{1}{2}\br{\dpd[2]{\phi}{u}\br{0, 0}u^2 + 2\dmd{\phi}{2}{u}{}{v}{}\br{0, 0}uv + \dpd[2]{\phi}{v}\br{0, 0}v^2} + R\br{u, v}, $$
where
$$ \lim_{\br{u, v} \to \br{0, 0}} \dfrac{R\br{u, v}}{u^2 + v^2} = 0. $$
Consider
\begin{align*}
\abr{\phi\br{u, v} - p, \N\br{p}}
& = \dfrac{1}{2}\abr{\dpd[2]{\phi}{u}\br{0, 0}u^2 + 2\dmd{\phi}{2}{u}{}{v}{}\br{0, 0}uv + \dpd[2]{\phi}{v}\br{0, 0}v^2, \N\br{p}} + \abr{R\br{u, v}, \N\br{p}} \\
& = \dfrac{1}{2}\A\br{w, w} + \abr{R\br{u, v}, \N\br{p}}, \qquad w = \dpd{\phi}{u}\br{0}u + \dpd{\phi}{v}\br{0}v.
\end{align*}
$ \A\br{w, w} $ is quadratic in $ \br{u, v} $ and $ R\br{u, v} $ is worse than quadratic in $ \br{u, v} $. So for $ u $ and $ v $ small, when we are close to $ p $, the function $ \abr{\phi\br{u, v} - p, \N\br{p}} $ has the same sign as $ \A\br{w, w} $. Write $ w = ax_1 + bx_2 $ where $ x_1 $ and $ x_2 $ are the principal directions then $ \A\br{w, w} = \lambda_1a^2 + \lambda_2b^2 $.
\begin{itemize}
\item $ \lambda_1, \lambda_2 > 0 $ gives $ \A\br{w, w} > 0 $.
\item $ \lambda_1 > 0 $ and $ \lambda_2 < 0 $ gives that $ \A\br{w, w} $ can be positive or negative.
\item $ \lambda_1, \lambda_2 < 0 $ gives $ \A\br{w, w} < 0 $.
\end{itemize}
\end{proof}

In fact we can do slightly better than this. Consider the chart map $ \phi : U \to S $. After a translation of $ S $, without loss of generality $ \phi\br{0, 0} = \br{0, 0, 0} $, after a rigid motion to $ S \subseteq \RR^3 $, without loss of generality
$$ \dpd{\phi}{u}\br{0, 0} = \br{1, 0, 0}, \qquad \dpd{\phi}{v}\br{0, 0} = \br{0, 1, 0}. $$
Taylor series implies that
\begin{align*}
\phi\br{u, v}
& = \br{0, 0, 0} + \br{\dpd{\phi}{u}\br{0, 0}u + \dpd{\phi}{v}\br{0, 0}v} + \dfrac{1}{2}\br{\dpd[2]{\phi}{u}\br{0, 0}u^2 + 2\dmd{\phi}{2}{u}{}{v}{}\br{0, 0}uv + \dpd[2]{\phi}{v}\br{0, 0}v^2} + \dots \\
& = \br{u, v, 0} + \dfrac{1}{2}\br{u^2\dpd[2]{\phi}{u}\br{0, 0} + 2uv\dmd{\phi}{2}{u}{}{v}{}\br{0, 0} + v^2\dpd[2]{\phi}{v}\br{0, 0}} + \dots.
\end{align*}

\pagebreak

Near $ \br{u, v} = \br{0, 0} $ the quadratic term satisfies
$$ \br{u^2\dpd[2]{\phi}{u}\br{0, 0} + 2uv\dmd{\phi}{2}{u}{}{v}{}\br{0, 0} + v^2\dpd[2]{\phi}{v}\br{0, 0}} \cdot \N\br{0, 0, 0} = \A\br{\br{u, v, 0}, \br{u, v, 0}} = \lambda_1u^2 + \lambda_2v^2, $$
because we chose $ \br{1, 0, 0} $ and $ \br{0, 1, 0} $ as our principal directions. So near $ \br{u, v} = \br{0, 0} $, the surface is approximated by the graph
$$ \phi\br{u, v} \approx \br{u, v, \dfrac{1}{2}\br{\lambda_1u^2 + \lambda_2v^2}}. $$

\lecture{17}{Monday}{12/11/18}

Say $ p \in S $ is
\begin{itemize}
\item \textbf{elliptic} if $ \K\br{p} > 0 $, so either $ \lambda_1, \lambda_2 > 0 $ or $ \lambda_1, \lambda_2 < 0 $,
\item \textbf{hyperbolic} if $ \K\br{p} < 0 $, so either $ \lambda_1 < 0 $ and $ \lambda_2 > 0 $ or vice versa,
\item \textbf{parabolic} if $ \K\br{p} = 0 $ and $ \H\br{p} \ne 0 $, so $ \lambda_1 = 0 $ and $ \lambda_2 \ne 0 $ or vice versa, and
\item \textbf{planar} if $ \K\br{p} = \H\br{p} = 0 $, so $ \lambda_1 = \lambda_2 = 0 $.
\end{itemize}

\begin{example*}
Consider the \textbf{Monkey saddle} with equation $ z = x^3 - 3x^2y $, so
$$ z = \Re \br{x + iy}^3 = \Re r^3e^{i3\theta} = r^3\cos 3\theta, $$
using polar coordinates $ x + iy = re^{i\theta} $. At $ \br{0, 0, 0} $, the principal curvatures are $ \lambda_1 = \lambda_2 = 0 $. So the previous theorem says, to second order near $ p = \br{0, 0, 0} $, the surface $ S $ looks like the plane $ z = 0 $. True, but not so informative. This has an umbilical point with $ \lambda_1 = \lambda_2 = 0 $ at $ \br{0, 0, 0} $, and hyperbolic points elsewhere.
\end{example*}

\subsection{How to compute Gaussian and mean curvature?}

We start with a simple observation.

\begin{lemma}
\label{lem:innerproduct}
Let $ V $ be a two-dimensional vector space with inner product $ \abr{\cdot, \cdot} $. Let $ e_1 $ and $ e_2 $ be an orthonormal basis for $ V $. Let $ x $ and $ y $ be elements of $ V $ and write $ x = x_1e_1 + x_2e_2 $ and $ y = y_1e_1 + y_2e_2 $. Then
$$ \abr{x, y} = \onebytwo{x_1}{x_2}\twobyone{y_1}{y_2}. $$
That is, we can compute inner products in $ V $ using the ordinary matrix multiplication between row vectors and column vectors, provided that we remember to use an orthonormal basis when writing elements of $ V $ as vectors.
\end{lemma}

\begin{proof}
Just write everything out, so
$$ \abr{x, y} = \abr{x_1e_1 + x_2e_2, y_1e_1 + y_2e_2} = x_1y_1\abr{e_1, e_1} + x_1y_2\abr{e_1, e_2} + x_2y_1\abr{e_2, e_1} + x_2y_2\abr{e_2, e_2} = x_1y_1 + x_2y_2, $$
as required. The final equality here uses the fact that $ e_1 $ and $ e_2 $ are orthonormal.
\end{proof}

\begin{remark*}
There is nothing special about the fact that $ V $ is two-dimensional here. The obvious generalisation of Lemma \ref{lem:innerproduct} holds for any finite-dimensional inner product space.
\end{remark*}

\begin{proposition}
\label{prop:computing}
Let $ \phi : U \to S $ be a chart on a regular surface $ S $. Define $ 2 \times 2 $ matrices
$$ \g = \twobytwo{\phi_u \cdot \phi_u}{\phi_u \cdot \phi_v}{\phi_v \cdot \phi_u}{\phi_v \cdot \phi_v}, \qquad \A = \twobytwo{\A\br{\phi_u, \phi_u}}{\A\br{\phi_u, \phi_v}}{\A\br{\phi_v, \phi_u}}{\A\br{\phi_v, \phi_v}}, \qquad \phi_u = \dpd{\phi}{u}, \qquad \phi_v = \dpd{\phi}{v}. $$
Then, writing $ \sigma = \g^{-1}\A $, we have
$$ \K = \det \sigma = \dfrac{\det \A}{\det \g}, \qquad \H = \dfrac{1}{2}\Tr \sigma. $$
\end{proposition}

\begin{remark*}
If $ \phi_u $ and $ \phi_v $ are the principal directions at $ p \in S $ then this is immediate, because in that case $ \g $ is the identity matrix and Proposition \ref{prop:computing} just restates the definitions of $ \K $ and $ \H $.
\end{remark*}

\pagebreak

\begin{proof}
Let $ p $ be a point on $ S $. Let us work in a basis for $ \T_p S $ given by the principal directions $ x_1\br{p} $ and $ x_2\br{p} $. Then, because this is an orthonormal basis, we can apply Lemma \ref{lem:innerproduct} to find,
$$ \A = \twobyone{\phi_u}{\phi_v}\onebytwo{-\d\N_p\br{\phi_u}}{-\d\N_p\br{\phi_v}}. $$
Here $ \br{\phi_u} $ is a row vector of size two and $ \br{-\d\N_p\br{\phi_u}} $ is a column vector of size two. Now write the matrix of $ \d\N_p : \T_p S \to \T_p S $ with respect to the basis $ x_1\br{p} $ and $ x_2\br{p} $ too,
$$ \A = \twobyone{\phi_u}{\phi_v}M\onebytwo{\phi_u}{\phi_v}, $$
where $ M $, the matrix of $ \d\N_p $ with respect to the basis $ x_1\br{p} $ and $ x_2\br{p} $, is a $ 2 \times 2 $ matrix. Then
$$ \det \A = \det \twobyone{\phi_u}{\phi_v}\det M\det \onebytwo{\phi_u}{\phi_v} = \det M\det \twobyone{\phi_u}{\phi_v}\onebytwo{\phi_u}{\phi_v} = \det M\det \g, $$
where for the last equality we used Lemma \ref{lem:innerproduct} again. Since $ \det M = \K $, by definition, this proves that $ \K = \det \A / \det \g $. Furthermore,
$$ \A = \twobyone{\phi_u}{\phi_v}\onebytwo{\phi_u}{\phi_v}\onebytwo{\phi_u}{\phi_v}^{-1}M\onebytwo{\phi_u}{\phi_v} = \g B, $$
where $ B = \onebytwo{\phi_u}{\phi_v}^{-1}M\onebytwo{\phi_u}{\phi_v} $ is conjugate to $ M $. Thus $ \sigma = \g^{-1}\A $ is conjugate to $ M $, and so $ \H = \Tr M = \Tr \sigma $ because conjugate matrices have the same trace.
\end{proof}

\lecture{18}{Tuesday}{13/11/18}

\begin{example*}
Let $ S $ be the surface in $ \RR^3 $ defined by the equation $ z = x^2 - y^2 $. Choose the chart
$$ \function[\phi]{U = \RR^2}{S}{\br{u, v}}{\br{u, v, u^2 - v^2}}. $$
Now $ \phi_u = \br{1, 0, 2u} $ and $ \phi_v = \br{0, 1, -2v} $, so
$$ \g = \twobytwo{\phi_u \cdot \phi_u}{\phi_u \cdot \phi_v}{\phi_v \cdot \phi_u}{\phi_v \cdot \phi_v} = \twobytwo{1 + 4u^2}{-4uv}{-4uv}{1 + 4v^2}. $$
Also,
$$ \phi_u \times \phi_v = \abs{\threebythree{i}{j}{k}{1}{0}{2u}{0}{1}{-2v}} = \br{-2u, 2v, 1}, $$
and so
$$ \N = \dfrac{\phi_u \times \phi_v}{\abs{\phi_u \times \phi_v}} = \br{\tfrac{-2u}{\sqrt{1 + 4u^2 + 4v^2}}, \tfrac{2v}{\sqrt{1 + 4u^2 + 4v^2}}, \tfrac{1}{\sqrt{1 + 4u^2 + 4v^2}}}. $$
Now $ \phi_{uu} = \br{1, 0, 2}, \phi_{uv} = \br{0, 0, 0}, \phi_{vv} = \br{1, 0, -2} $, so
$$ \A = \twobytwo{\N \cdot \phi_{uu}}{\N \cdot \phi_{uv}}{\N \cdot \phi_{vu}}{\N \cdot \phi_{vv}} = \twobytwo{\tfrac{2}{\sqrt{1 + 4u^2 + 4v^2}}}{0}{0}{\tfrac{-2}{\sqrt{1 + 4u^2 + 4v^2}}} = \dfrac{1}{\sqrt{1 + 4u^2 + 4v^2}}\twobytwo{2}{0}{0}{-2}. $$
Then
$$ \sigma = \g^{-1}\A = \dfrac{1}{1 + 4u^2 + 4v^2}\twobytwo{1 + 4v^2}{4uv}{4uv}{1 + 4u^2} \cdot \A = \dfrac{2}{\sqrt{1 + 4u + 4v^2}^3}\twobytwo{1 + 4v^2}{-4uv}{4uv}{-1 - 4u^2}. $$
So
$$ \K = \det \sigma = 4\br{\dfrac{-\br{1 + 4u^2}\br{1 + 4v^2} + 16u^2v^2}{\br{1 + 4u^2 + 4v^2}^3}} = \dfrac{-4}{\br{1 + 4u^2 + 4v^2}^2}, $$
and
$$ \H = \dfrac{1}{2}\Tr \sigma = \dfrac{4v^2 - 4u^2}{\sqrt{1 + 4u^2 + 4v^2}^3}. $$
\end{example*}

\pagebreak

\subsection{The first fundamental form}

\begin{definition}
Let $ S \subset \RR^3 $ be a regular surface, and let $ p \in S $. The \textbf{first fundamental form} at $ p $, also called the \textbf{metric} at $ p $, is the bilinear map
$$ \function[\g]{\T_p S \times \T_p S}{\RR}{\br{v, w}}{\abr{v, w}}. $$
\end{definition}

Then $ \g $ is symmetric, bilinear, and non-degenerate. If $ \phi : U \to S $ is a chart near $ p \in S $, then $ \phi_u $ and $ \phi_v $ are a basis for $ \T_p S $ and in this basis, $ \g $ has the form
\begin{align*}
\g\br{a\phi_u + b\phi_v, c\phi_u + d\phi_v}
& = ac\g\br{\phi_u, \phi_v} + ad\g\br{\phi_u, \phi_v} + bc\g\br{\phi_u, \phi_v} + bd\g\br{\phi_u, \phi_v} \\
& = \onebytwo{a}{b}\twobytwo{\g\br{\phi_u, \phi_u}}{\g\br{\phi_u, \phi_v}}{\g\br{\phi_v, \phi_u}}{\g\br{\phi_v, \phi_v}}\twobyone{c}{d}
= \onebytwo{a}{b}\g\twobyone{c}{d},
\end{align*}
that is the metric is represented by the symmetric matrix that we called $ \g $ before. Metric determines arc-length. If $ \alpha : \sbr{a, b} \to S $ is a curve contained in $ \phi\br{U} $, so that $ \alpha\br{t} = \phi\br{u\br{t}, v\br{t}} $ for some smooth functions $ u, v : \sbr{a, b} \to \RR $, then
$$ \L\br{\alpha} = \intd{a}{b}{\abs{\alpha'\br{t}}}{t} = \intd{a}{b}{\abs{\tpd{\phi}{u}\tod{u}{t} + \tpd{\phi}{v}\tod{v}{t}}}{t} = \intd{a}{b}{\sqrt{\onebytwo{\tod{u}{t}}{\tod{v}{t}} \cdot \g \cdot \twobyone{\tod{u}{t}}{\tod{v}{t}}}}{t}. $$

\begin{definition}
A smooth map $ F : S_1 \to S_2 $ is called a \textbf{local isometry} if it preserves the first fundamental form, so
$$ \abr{\d F_p\br{x}, \d F_p\br{y}} = \abr{x, y}, \qquad x, y \in \T_p S_1, \qquad p \in S_1. $$
$ F $ is an \textbf{isometry} if it is also bijective.
\end{definition}

\begin{proposition}
Local isometries are local diffeomorphisms.
\end{proposition}

\lecture{19}{Friday}{16/11/18}

\begin{proof}
If $ F $ is a local isometry then, for each $ p \in S $, $ \d F_p $ is injective. Suppose not. Then there exists $ 0 \ne v \in \T_p S $ such that $ \d F_p\br{v} = 0 $, so $ \abr{\d F_p\br{v}, \d F_p\br{w}} = 0 $ for all $ w \in \T_p S $, so $ \abr{v, w} = 0 $ for all $ w \in \T_p S $, so $ v = 0 $, a contradiction. Now $ \d F_p : \T_p S_1 \to \T_{F\br{p}} S_2 $ is linear and injective, and $ \dim \T_p S_1 = \dim \T_p S_2 = 2 $, so $ \d F_p $ is an isomorphism. From before, since $ \d F_p $ is an isomorphism, $ F $ is a local diffeomorphism, that is, there is an open neighbourhood $ U $ of $ p $ in $ S_1 $ such that $ \eval{F}_U : U \to F\br{U} $ is a diffeomorphism.
\end{proof}

\begin{proposition}
A smooth map $ F : S_1 \to S_2 $ between regular surfaces is a local isometry if and only if it preserves lengths of curves, that is if and only if for all $ \alpha : \sbr{a, b} \to S_1 $ smooth we have
$$ \L\br{\alpha} = \L\br{F \circ \alpha}. $$
\end{proposition}

\begin{proof}
\hfill
\begin{itemize}
\item[$ \implies $] Suppose that $ F $ is a local isometry. Then
\begin{align*}
\L\br{F \circ \alpha}
& = \intd{a}{b}{\abs{\br{F \circ \alpha}'\br{t}}}{t}
= \intd{a}{b}{\sqrt{\abr{\br{F \circ \alpha}'\br{t}, \br{F \circ \alpha}'\br{t}}}}{t} \\
& = \intd{a}{b}{\sqrt{\abr{\d F_{\alpha\br{t}}\br{\alpha'\br{t}}, \d F_{\alpha\br{t}}\br{\alpha'\br{t}}}}}{t}
= \intd{a}{b}{\sqrt{\abr{\alpha'\br{t}, \alpha'\br{t}}}}{t} \\
& = \intd{a}{b}{\abs{\alpha'\br{t}}}{t}
= \L\br{\alpha},
\end{align*}
by definition of $ \d F_{\alpha\br{t}} $ and as $ F $ is a local isometry.
\item[$ \impliedby $] Given $ v \in \T_p S_1 $ and a curve $ \alpha : \br{-\epsilon, \epsilon} \to S_1 $ with $ \alpha\br{0} = p $ and $ \alpha'\br{0} = v $. Consider the curve $ \eval{\alpha}_{\sbr{-\delta, t}} : \sbr{-\delta, t} \to S_1 $, so
$$
\begin{tikzcd}
& p \in S_1 \arrow{r}{F} & S_2 \\
\br{-\epsilon, \epsilon} \arrow{ur}{\alpha} & \sbr{-\delta, t} \arrow[subset]{l} \arrow{u}{\eval{\alpha}_{\sbr{-\delta, t}}} \arrow{ur} &
\end{tikzcd}.
$$

\pagebreak

Know $ \L\br{\eval{\alpha}_{\sbr{-\delta, t}}} = \L\br{\eval{\br{F \circ \alpha}}_{\sbr{-\delta, t}}} $, that is
$$ \intd{-\delta}{t}{\abs{\alpha'\br{s}}}{s} = \intd{-\delta}{t}{\abs{\d F_{\alpha\br{s}}\br{\alpha'\br{s}}}}{s}. $$
Taking $ \tod{}{t} $ and setting $ t = 0 $, $ \abs{\alpha'\br{0}} = \abs{\d F_p\br{\alpha'\br{0}}} $, that is
$$ \abr{v, v} = \abr{\d F_p\br{v}, \d F_p\br{v}}, \qquad v \in \T_p S_1, \qquad p \in S_1. $$
Now,
$$ \abr{X + Y, X + Y} = \abr{X, X} + \abr{X, Y} + \abr{Y, X} + \abr{Y, Y} = \abr{X, X} + 2\abr{X, Y} + \abr{Y, Y}, $$
so
$$ \abr{X, Y} = \dfrac{1}{2}\br{\abr{X + Y, X + Y} - \abr{X, X} - \abr{Y, Y}}. $$
Set $ X = \d F_p\br{v} $ and $ Y = \d F_p\br{w} $, so
\begin{align*}
\abr{\d F_p\br{v}, \d F_p\br{w}}
& = \dfrac{1}{2}\br{\abr{\d F_p\br{v + w}, \d F_p\br{v + w}} - \abr{\d F_p\br{v}, \d F_p\br{v}} - \abr{\d F_p\br{w}, \d F_p\br{w}}} \\
& = \dfrac{1}{2}\br{\abr{v + w, v + w} - \abr{v, v} - \abr{w, w}}
= \abr{v, w}.
\end{align*}
Therefore $ F $ is a local isometry.
\end{itemize}
\end{proof}

\begin{example*}
Let $ S \subset \RR^3 $ be any regular surface. Let $ B \in \SO\br{3} $, $ 3 \times 3 $ matrices $ B $ such that $ B^\intercal B = \id $ and $ \det B = 1 $. Set
$$ S' = \cbr{B\vec{x} \st \vec{x} \in S}. $$
Then
$$ \function[B]{S}{S'}{\vec{x}}{B\vec{x}} $$
is bijective. Let $ p \in S $ and $ v, w \in \T_p S $ be arbitrary. Then $ \d F_p\br{v} = Bv $ and
$$ \abr{\d F_p\br{v}, \d F_p\br{w}} = \abr{Bv, Bw} = \br{Bv}^\intercal Bw = v^\intercal B^\intercal Bw = v^\intercal w = \abr{v, w}, $$
so $ B $ is a local isometry and an isometry.
\end{example*}

\begin{example*}
Let
$$ S_1 = \cbr{\br{x, y, 0} \in \RR^3 \st x \in \RR, \ y \in \RR}, \qquad S_2 = \cbr{\br{x, y, z} \in \RR^3 \st x^2 + y^2 = 1}. $$
Define
$$ \function[F]{S_1}{S_2}{\br{u, v, 0}}{\br{\cos u, \sin u, v}}. $$
At $ p = \br{u, v, 0} \in S_1 $ a tangent vector $ \vec{v} = \br{a, b, 0} $ can be written as $ \alpha'\br{0} $ where $ \alpha\br{t} = \br{u + at, v + bt, 0} $. Now
$$ \br{F \circ \alpha}\br{t} = \br{\cos \br{u + at}, \sin \br{u + at}, v + bt}, $$
so $ \br{F \circ \alpha}'\br{0} = \br{-a\sin u, a\cos u, b} $. Then
$$ \abr{\d F_p\br{\vec{v}}, \d F_p\br{\vec{v}}} = \abr{\br{-a\sin u, a\cos u, b}, \br{-a\sin u, a\cos u, b}} = a^2 + b^2 = \abr{\vec{v}, \vec{v}}, \qquad \vec{v} \in \T_p S_1, $$
and so, as before,
$$ \abr{\vec{v}, \vec{w}} = \abr{\d F_p\br{\vec{v}}, \d F_p\br{\vec{w}}}, \qquad \vec{v}, \vec{w} \in \T_p S_1. $$
Thus $ F $ is a local isometry. It is not an isometry because it is not bijective.
\end{example*}

\pagebreak

\subsection{Christoffel symbols}

\lecture{20}{Monday}{19/11/18}

Now, $ \tod{\phi}{x_1} $ and $ \tod{\phi}{x_2} $ span $ \T_p S $ and $ \N\br{p} $ is orthogonal to $ \T_p S $ so $ \tod{\phi}{x_1} $, $ \tod{\phi}{x_2} $, and $ \N $ form a basis for $ \RR^3 $. Write
\begin{equation}
\label{eq:2}
\dmd{\phi}{2}{x_i}{}{x_j}{} = \Gamma_{ij}^1\dpd{\phi}{x_1} + \Gamma_{ij}^2\dpd{\phi}{x_2} + \A_{ij}\N,
\end{equation}
for some coefficients $ \Gamma_{ij}^k $ and $ \A_{ij} $. Then
$$ \dmd{\phi}{2}{x_i}{}{x_j}{} = \dmd{\phi}{2}{x_j}{}{x_i}{}, $$
so $ \Gamma_{ij}^k = \Gamma_{ji}^k $ for all $ i, j, k $.

\begin{definition}
The $ \Gamma_{ij}^k $ are called \textbf{Christoffel symbols}.
\end{definition}

Taking $ \br{\ref{eq:2}} \cdot \N $, we get
$$ \dmd{\phi}{2}{x_i}{}{x_j}{} \cdot \N = \A_{ij}. $$

\begin{note*}
$ \nabla_ie_j = \Gamma_{ij}^ke_k $, where $ \nabla_i $ is the metric connection in the local frame $ e_j $.
\end{note*}

\begin{proposition}
The Christoffel symbols are determined only by the first fundamental form, or metric.
\end{proposition}

\begin{proof}
From $ \br{\ref{eq:2}} \cdot \tod{\phi}{x_k} $,
$$ \dmd{\phi}{2}{x_i}{}{x_j}{} \cdot \dpd{\phi}{x_k} = \Gamma_{ij}^1\g_{1k} + \Gamma_{ij}^2\g_{2k}, \qquad \g = \twobytwo{\g_{11}}{\g_{12}}{\g_{21}}{\g_{22}}. $$
Also,
$$ \dpd{}{x_i}\br{\dpd{\phi}{x_j} \cdot \dpd{\phi}{x_k}} = \dmd{\phi}{2}{x_i}{}{x_j}{} \cdot \dpd{\phi}{x_k} + \dpd{\phi}{x_j} \cdot \dmd{\phi}{2}{x_i}{}{x_k}{}. $$
Taking $ j = k $,
$$ \dpd{}{x_i}\br{\g_{jj}} = 2\br{\Gamma_{ij}^1\g_{1j} + \Gamma_{ij}^2\g_{2j}}. $$
If $ j \ne k $, then either $ i = j $ or $ i = k $. Taking $ i = j \ne k $,
$$ \dpd{}{x_i}\br{\g_{ik}} = \dmd{\phi}{2}{x_i}{}{x_i}{} \cdot \dpd{\phi}{x_k} + \dpd{\phi}{x_i} \cdot \dmd{\phi}{2}{x_i}{}{x_k}{} = \dmd{\phi}{2}{x_i}{}{x_i}{} \cdot \dpd{\phi}{x_k} + \dfrac{1}{2}\dpd{}{x_k}\br{\g_{ii}}. $$
Rewriting,
$$ \Gamma_{ij}^1\g_{1k} + \Gamma_{ij}^2\g_{2k} = \dmd{\phi}{2}{x_i}{}{x_i}{} \cdot \dpd{\phi}{x_k} = \dpd{}{x_i}\br{\g_{ik}} - \dfrac{1}{2}\dpd{}{x_k}\br{\g_{ii}}. $$
We have lots of equations involving $ \Gamma_{ij}^k $, $ \g_{ij} $, and derivatives of $ \g_{ij} $. In fact we can solve these for $ \Gamma_{ij}^k $. Then $ i = j = 1 $ and $ k = 1, 2 $ gives
$$ \Gamma_{11}^1\g_{11} + \Gamma_{11}^2\g_{21} = \dfrac{1}{2}\dpd{\g_{11}}{x_1}, \qquad \Gamma_{11}^1\g_{12} + \Gamma_{11}^2\g_{22} = \dpd{\g_{12}}{x_1} - \dfrac{1}{2}\dpd{\g_{11}}{x_2}. $$
That is,
$$ \g\twobyone{\Gamma_{11}^1}{\Gamma_{11}^2} = \twobyone{\tfrac{1}{2}\tpd{\g_{11}}{x_1}}{\tpd{\g_{12}}{x_1} - \tfrac{1}{2}\tpd{\g_{11}}{x_2}}. $$
Then $ \g $ is invertible, because $ \tod{\phi}{x_1} $ and $ \tod{\phi}{x_2} $ are linearly independent. Similarly,
$$ \g\twobyone{\Gamma_{21}^1}{\Gamma_{21}^2} = \twobyone{\tfrac{1}{2}\tpd{\g_{11}}{x_2}}{\tfrac{1}{2}\tpd{\g_{22}}{x_1}}, \qquad \g\twobyone{\Gamma_{22}^1}{\Gamma_{22}^2} = \twobyone{\tpd{\g_{21}}{x_2} - \tfrac{1}{2}\tpd{\g_{22}}{x_1}}{\tfrac{1}{2}\tpd{\g_{22}}{x_2}}. $$
Thus can solve these equations to write all $ \Gamma_{ij}^k $ in terms of $ \g_{ij} $'s and $ \tpd{\g_{jk}}{x_i} $'s.
\end{proof}

\pagebreak

\subsection{Theorema Egregium}

\lecture{21}{Tuesday}{20/11/18}

Recall that $ \K = \det \A_{ij} / \det \g_{ij} $. The goal is that $ \K $ depends only on $ \g $.

\begin{notation*}
Set $ x_1 = u $ and $ x_2 = v $. Write $ \phi_u = \tod{\phi}{x_1} = \tod{\phi}{u} $, without confusion with indices.
\end{notation*}

Mixed partial derivatives are equal, so
$$ \dpd{}{x_2}\br{\dmd{\phi}{2}{x_1}{}{x_1}{}} = \dpd{}{x_1}\br{\dmd{\phi}{2}{x_1}{}{x_2}{}}, $$
so
$$ \dpd{}{x_2}\br{\Gamma_{11}^1\dpd{\phi}{x_1} + \Gamma_{11}^2\dpd{\phi}{x_2} + \A_{11}\N} = \dpd{}{x_1}\br{\Gamma_{12}^1\dpd{\phi}{x_1} + \Gamma_{12}^2\dpd{\phi}{x_2} + \A_{12}\N}. $$
Thus
\begin{equation}
\label{eq:3}
\Gamma_{11}^1\phi_{uv} + \Gamma_{11v}^1\phi_u + \Gamma_{11}^2\phi_{vv} + \Gamma_{11v}^2\phi_v + \A_{11v}\N + \A_{11}\N_v = \Gamma_{12}^1\phi_{uu} + \Gamma_{12u}^1\phi_u + \Gamma_{12}^2\phi_{vu} + \Gamma_{12u}^2\phi_v + \A_{12u}\N + \A_{12}\N_u.
\end{equation}
Will take the $ \phi_u $-component of $ \br{\ref{eq:3}} $, and use
$$ \phi_{uv} = \Gamma_{12}^1\phi_u + \Gamma_{12}^2\phi_v + \A_{12}\N, \qquad \phi_{vv} = \Gamma_{22}^1\phi_u + \Gamma_{22}^2\phi_v + \A_{22}\N, \qquad \Gamma_{uu} = \Gamma_{11}^1\phi_u + \Gamma_{11}^2\phi_v + \A_{11}\N, $$
$$ \A_{11} = -\dpd{\phi}{u} \cdot \N_u, \qquad \A_{21} = -\dpd{\phi}{v} \cdot \N_u, \qquad \A_{12} = -\dpd{\phi}{u} \cdot \N_v, \qquad \A_{22} = -\dpd{\phi}{v} \cdot \N_v. $$
Now
$$ \N_u = a\phi_u + b\phi_v, \qquad \N_v = c\phi_u + d\phi_v, $$
since $ \N \cdot \N = 1 $, so $ \N_u \cdot \N + \N \cdot \N_u = 0 $ and $ \N_v \cdot \N + \N \cdot \N_v = 0 $, so $ \N_u, \N_v \in \T_p S $. Then
$$ -\A_{11} = a\g_{11} + b\g_{12}, \qquad -\A_{21} = a\g_{21} + b\g_{22}, \qquad -\A_{12} = c\g_{11} + d\g_{12}, \qquad -\A_{22} = c\g_{21} + d\g_{22}, $$
so
$$ -\A = \g\twobytwo{a}{c}{b}{d}, $$
so
$$ \twobytwo{a}{c}{b}{d} = -\g^{-1}\A = \dfrac{-1}{\g_{11}\g_{12} - \g_{12}^2}\twobytwo{\g_{22}}{-\g_{12}}{-\g_{21}}{\g_{11}}\twobytwo{\A_{11}}{\A_{12}}{\A_{21}}{\A_{22}}, $$
so
$$ a = \dfrac{\g_{12}\A_{21} - \g_{22}\A_{11}}{\g_{11}\g_{22} - \g_{12}^2}, \qquad c = \dfrac{\g_{12}\A_{22} - \g_{22}\A_{12}}{\g_{11}\g_{22} - \g_{12}^2}. $$
Now, take $ \br{\ref{eq:3}} $, dot with $ \phi_u $, and use these equations above. We find,
$$ \Gamma_{11}^1\Gamma_{12}^1 + \Gamma_{11v}^1 + \Gamma_{11}^2\Gamma_{22}^1 + \A_{11}\br{\dfrac{\g_{12}\A_{22} - \g_{22}\A_{12}}{\det \g}} = \Gamma_{12}^1\Gamma_{11}^1 + \Gamma_{12u}^1 + \Gamma_{12}^2\Gamma_{12}^1 + \A_{12}\br{\dfrac{\g_{12}\A_{21} - \g_{22}\A_{11}}{\det \g}}. $$
Rearranging,
$$ \Gamma_{11v}^1 - \Gamma_{12u}^1 = \Gamma_{12}^2\Gamma_{12}^1 - \Gamma_{11}^2\Gamma_{22}^1 + \dfrac{\br{\A_{12}\A_{21} - \A_{11}\A_{22}}\g_{12}}{\det \g}. $$
In other words,
$$ \K \g_{12} = \Gamma_{12}^2\Gamma_{12}^1 - \Gamma_{11}^2\Gamma_{22}^1 + \Gamma_{12u}^1 - \Gamma_{11v}^1. $$
Similarly, taking the $ \phi_v $ component of $ \br{\ref{eq:3}} $
$$ \K \g_{11} = \Gamma_{11}^1\Gamma_{21}^2 + \Gamma_{11}^2\Gamma_{22}^2 - \Gamma_{12}^1\Gamma_{11}^2 - \Gamma_{12}^2\Gamma_{12}^2 + \Gamma_{11v}^2 - \Gamma_{12u}^2. $$
These are the \textbf{Gauss equations}.

\begin{note*}
$ \K \g_{12} $ and $ \K \g_{11} $ depend only on $ \g $. But $ \g $ is invertible, so $ \g_{11} $ and $ \g_{12} $ cannot simultaneously vanish.
\end{note*}

We have proved the following.

\begin{theorem}[Theorema Egregium, Gauss]
The Gaussian curvature $ \K $ depends only on the first fundamental form.
\end{theorem}

In other words, it is an intrinsic invariant of the surface $ \br{S, \g} $, as a Riemannian manifold.

\pagebreak

\begin{corollary}
If $ f : S_1 \to S_2 $ is a local isometry then $ \K_{S_1}\br{p} = \K_{S_2}\br{f\br{p}} $ for all $ p \in S_1 $.
\end{corollary}

\begin{corollary}
There is no local isometry from a plane to a sphere.
\end{corollary}

\begin{proof}
A plane has $ \K = 0 $ everywhere and a sphere has $ \K = 1 / r^2 $ everywhere, where $ r $ is the radius.
\end{proof}

In particular, there is no way to draw a map of the Earth on a flat sheet of paper without distorting lengths or angles. This is also why it is hard to gift-wrap a ball.

\lecture{22}{Friday}{23/11/18}

\begin{theorem}
Let $ S $ be a compact surface with $ \K > 0 $, such that $ \K $ is constant, then $ S $ is umbilical.
\end{theorem}

\begin{proof}
Let $ x \in S $, and $ \lambda_1\br{x} \le \lambda_2\br{x} $ be principal curvatures at $ x $. The aim is that $ \lambda_1\br{x} = \lambda_2\br{x} $. The idea of the proof is to assume $ \lambda_1 < \lambda_2 $ for a contradiction. Let $ p \in S $ be a point maximising $ \lambda_2\br{x} $, so minimises $ \lambda_1\br{x} $ and
$$ \lambda_1\br{p} \le \lambda_1\br{x} \le \lambda_2\br{x} \le \lambda_2\br{p}. $$
The aim is if $ \lambda_1\br{p} = \lambda_2\br{p} $, then $ \lambda_1\br{x} = \lambda\br{x} $, for all $ x \in S $. Assume $ \lambda_1\br{p} < \lambda_2\br{p} $, and want to get a contradiction. After a rigid motion, around $ p = \br{0, 0, 0} $,
$$ \phi\br{u, v} = \br{u, v, F\br{u, v}} + \dots = \dfrac{1}{2}\br{\lambda_1u^2 + \lambda_2v^2} + \dots, $$
by Taylor expansion. At $ \br{0, 0} $, $ F\br{0, 0} = F_u\br{0, 0} = F_v\br{0, 0} = 0 $, so
$$ \dpd{\phi}{u}\br{0, 0} = \br{1, 0, F_u\br{0, 0}} = \br{1, 0, 0}, \qquad \dpd{\phi}{u}\br{0, 0} = \br{0, 1, F_v\br{0, 0}} = \br{0, 1, 0}, $$
so
$$ \N = \dfrac{\phi_u \times \phi_v}{\abs{\phi_u \times \phi_v}} = \dfrac{\br{-F_u, -F_v, 1}}{\sqrt{1 + \abs{\nabla F}^2}} = \br{0, 0, 1}, $$
at $ \br{0, 0} $. Then
$$ \A = \twobytwo{F_{uu}}{F_{uv}}{F_{vu}}{F_{vv}} = \twobytwo{\lambda_1}{0}{0}{\lambda_2}, \qquad \g = \twobytwo{\phi_u \cdot \phi_u}{\phi_u \cdot \phi_v}{\phi_v \cdot \phi_u}{\phi_v \cdot \phi_v} = \twobytwo{1}{0}{0}{1}. $$
Thus $ \K = \det \A / \det \g = \lambda_1\lambda_2 $. Let
$$ E_1 = \dfrac{\br{1, 0, F_u}}{\sqrt{1 + F_u^2}}, \qquad E_2 = \dfrac{\br{0, 1, F_v}}{\sqrt{1 + F_v^2}}, $$
and let
$$ h_1\br{t} = \A\br{E_1\br{0, t}, E_1\br{0, t}}, \qquad h_2\br{t} = \A\br{E_2\br{t, 0}, E_2\br{t, 0}}. $$
Check \footnote{Exercise}
$$ h_1\br{t} = \eval{\dfrac{1}{1 + F_u^2} \cdot \dfrac{F_{uu}}{\sqrt{1 + \abs{\nabla F}^2}}}_{\br{0, t}}, \qquad h_2\br{t} = \eval{\dfrac{1}{1 + F_v^2} \cdot \dfrac{F_{vv}}{\sqrt{1 + \abs{\nabla F}^2}}}_{\br{t, 0}}. $$
Recall that
$$ \lambda_1\br{p} \le \lambda_1\br{\phi\br{0, t}} = \min\cbr{\A\br{v, v} \st v \in \T_{\phi\br{0, t}} S, \ \abs{v} = 1} \le \A\br{E_1\br{0, t}, E_1\br{0, t}} = h_1\br{t}, $$
and
$$ \lambda_2\br{p} \ge \lambda_2\br{\phi\br{t, 0}} = \max\cbr{\A\br{v, v} \st v \in \T_{\phi\br{t, 0}} S, \ \abs{v} = 1} \ge \A\br{E_2\br{t, 0}, E_2\br{t, 0}} = h_2\br{t}, $$
so $ h_1\br{t} $ is minimum at $ t = 0 $ and $ h_2\br{t} $ is maximum at $ t = 0 $, so $ h_1''\br{0} \ge 0 $ and $ h_2''\br{0} \le 0 $, so
$$ h_1''\br{0} - h_2''\br{0} \ge 0. $$
Assume $ \lambda_1 < \lambda_2 $. Thus $ \lambda_1\lambda_2\br{\lambda_1 - \lambda_2} < 0 $, \footnote{Exercise} a contradiction.
\end{proof}

\begin{corollary}
Let $ \K > 0 $ be constant. Then $ S $ is compact and connected implies that $ S = \S^2 $.
\end{corollary}

\pagebreak

\section{Area of surfaces}

\subsection{Area}

\lecture{23}{Monday}{26/11/18}

Let $ S \subset \RR^3 $ be a regular surface and $ \phi : U \to S $ be a chart. How to measure the area on $ S $? Consider a tiny rectangle in $ U $ with sides $ \br{\delta u, 0} $ and $ \br{0, \delta v} $. The area of this, in $ U $, is $ \delta u\delta v $. If $ \delta u $ and $ \delta v $ are small then the area of the image of this rectangle in $ \phi\br{U} $ is approximately the area of the parallelogram in $ \RR^3 $ with sides $ \br{\tod{\phi}{u}}\delta u $ and $ \br{\tod{\phi}{v}}\delta v $, that is
$$ \abs{\br{\dpd{\phi}{u}\delta u} \times \br{\dpd{\phi}{v}\delta v}} = \abs{\dpd{\phi}{u} \times \dpd{\phi}{v}}\delta u\delta v. $$

\begin{definition}
If $ D \subset U $ is compact, then
$$ \area \phi\br{D} = \iintd{D}{\abs{\tpd{\phi}{u} \times \tpd{\phi}{v}}}{u}{v}. $$
\end{definition}

Need to show this integral
\begin{itemize}
\item converges, and
\item is independent of choice of chart.
\end{itemize}

\begin{proposition}
This definition does not depend on the choice of coordinate chart $ \phi : U \to S $.
\end{proposition}

\begin{proof}
Let $ \phi : U \to S $ and $ \psi : U' \to S $ be charts whose images contain $ \phi\br{D} = \psi\br{D} $. After shrinking $ U $ and $ U' $ if necessary, we have that $ f = \phi^{-1} \circ \psi : U' \to U $ is well-defined and smooth. Write
$$ f\br{u, v} = \br{x\br{u, v}, y\br{u, v}}, $$
so
$$
\begin{tikzcd}
D \subset U \arrow{r}{\phi} & S \\
D' \subset U' \arrow{u}{f} \arrow{ur}[swap]{\psi} &
\end{tikzcd}.
$$
Then $ \psi = \phi \circ f $ so
$$ \dpd{\psi}{u} \times \dpd{\psi}{v} = \br{\dpd{\phi}{x}\dpd{x}{u} + \dpd{\phi}{y}\dpd{y}{u}} \times \br{\dpd{\phi}{x}\dpd{x}{v} + \dpd{\phi}{y}\dpd{y}{v}} = \Delta\br{\dpd{\phi}{x} \times \dpd{\phi}{y}}, \qquad \Delta\br{u, v} = \dpd{x}{u}\dpd{y}{v} - \dpd{y}{u}\dpd{x}{v}. $$
So
\begin{align*}
\area \psi\br{D'}
& = \iintd{D'}{\abs{\tpd{\psi}{u} \times \tpd{\psi}{v}}}{u}{v}
= \iintd{D'}{\abs{\tpd{\phi}{u} \times \tpd{\phi}{v}}\abs{\det \twobytwo{\tpd{x}{u}}{\tpd{y}{u}}{\tpd{x}{v}}{\tpd{y}{v}}}}{u}{v} \\
& = \iintd{D}{\abs{\tpd{\phi}{x} \times \tpd{\phi}{y}}}{x}{y}
= \area \phi\br{D},
\end{align*}
by the change of variable formula.
\end{proof}

\begin{note*}
$ S \subset \RR^3 $, and $ \RR^3 $ has a distinguished $ 3 $-form $ \d x \wedge \d y \wedge \d z $. If $ S $ is oriented, so we have chosen a normal vector field $ \N $ on $ S $, then can consider the $ 2 $-form $ \omega = 2_\N\br{\d x \wedge \d y \wedge \d z} $, where $ 2_\N $ is the contraction, and measure areas on $ S $ using this $ 2 $-form. Up to sign, this gives the same notion of area.
\end{note*}

\pagebreak

\begin{lemma}
$$ \area \phi\br{D} = \iintd{D}{\sqrt{\det \g}}{u}{v}. $$
\end{lemma}

\begin{proof}
$ \abs{a \times b}^2 = \abs{a}^2\abs{b}^2 - \br{a \cdot b}^2 $, so
$$ \abs{\dpd{\phi}{u} \times \dpd{\phi}{v}}^2 = \abs{\dpd{\phi}{u}}^2\abs{\dpd{\phi}{v}}^2 - \br{\dpd{\phi}{u} \cdot \dpd{\phi}{v}}^2. $$
But
$$ \g = \twobytwo{\tpd{\phi}{u} \cdot \tpd{\phi}{u}}{\tpd{\phi}{u} \cdot \tpd{\phi}{v}}{\tpd{\phi}{v} \cdot \tpd{\phi}{u}}{\tpd{\phi}{v} \cdot \tpd{\phi}{v}}, $$
so
$$ \abs{\dpd{\phi}{u} \times \dpd{\phi}{v}}^2 = \g_{11}\g_{22} - \g_{21}^2 = \det \g. $$
Thus
$$ \area \phi\br{D} = \iintd{D}{\abs{\tpd{\phi}{u} \times \tpd{\phi}{v}}}{u}{v} = \iintd{D}{\sqrt{\det \g}}{u}{v}. $$
\end{proof}

\begin{definition}
Assume $ S $ is compact. If $ f : S \to \RR $ is a smooth function on $ S $, $ \phi : U \to S $ is a chart, and $ D \subset U $ is a compact subset, then we define
$$ \intd{\phi\br{D}}{}{f}{A} = \iintd{D}{\br{f \circ \phi}\br{u, v}\abs{\tpd{\phi}{u} \times \tpd{\phi}{v}}}{u}{v} = \iintd{D}{\br{f \circ \phi}\sqrt{\det \g}}{u}{v}. $$
\end{definition}

This does not depend on chart $ \phi : U \to S $. To integrate over all of $ S $ we divide $ S $ into pieces
$$ S = S_1 \cup \dots \cup S_k, \qquad S_i \cap S_j \subset \br{\partial S_i} \cap \br{\partial S_j}, \qquad i \ne j, $$
such that for each $ i $ there exists a chart $ \phi : U_i \to S $ such that $ S_i \subset \phi\br{U_i} $ and $ S_i = \phi\br{D_i} $ for some compact set $ D_i \subset U_i $. By compactness of $ S $, such a partition $ S = S_1 \cup \dots \cup S_k $ always exists.

\begin{definition}
Define
$$ \intd{S}{}{f}{A} = \sum_{i = 1}^k \intd{S_i}{}{f}{A}. $$
\end{definition}

Claim that this is independent of the choice of decomposition.

\lecture{24}{Tuesday}{27/11/18}

\begin{example*}
What is the surface area of the unit sphere in $ \RR^3 $? Use spherical polar coordinates
$$ \psi\br{\theta, \phi} = \br{\cos \theta\sin \phi, \sin \theta\sin \phi, \cos \phi}, \qquad \epsilon \le \theta \le 2\pi - \epsilon, \qquad \epsilon \le \phi \le \pi - \epsilon. $$
Are the columns of the matrix
$$ \onebytwo{\psi_\theta}{\psi_\phi} = \threebyone{-\sin \theta\sin \phi & \cos \theta\cos \phi}{\cos \theta\sin \phi & \sin \theta\cos \phi}{0 & -\sin \phi}. $$
linearly independent? If $ \sin \phi \ne 0 $ then the first two rows of the first column is not both zero and the last row of the second column is not zero. So this surface is regular, because $ \sin \phi \ne 0 $ in our chart, since we took $ \phi \in \br{0, \pi} $. Then
\begin{align*}
\area \psi\br{D}
& = \iintd{D}{\abs{\psi_\theta \times \psi_\phi}}{\theta}{\phi}
= \intd{\epsilon}{\pi - \epsilon}{\intd{\epsilon}{2\pi - \epsilon}{\sin \phi}{\theta}}{\phi} \\
& = \br{2\pi - 2\epsilon}\sbr{\cos \phi}_{\pi - \epsilon}^\epsilon
= \br{2\pi - 2\epsilon}\br{\cos \epsilon - \cos \br{\pi - \epsilon}},
\end{align*}
so
$$ \lim_{\epsilon \to 0} \area \psi\br{D} = 4\pi. $$
Compute the area of the rest by using another chart, which tends to zero as $ \epsilon \to 0 $. Thus $ \area \S^2 = 4\pi $.
\end{example*}

\pagebreak

\section{Geodesics}

\subsection{Geodesic curvature}

Geodesics are the shortest paths. Let $ \gamma : \sbr{a, b} \to S $ be a regular curve in a regular surface $ S $. Suppose that $ S $ is oriented, with normal vector field $ \N $, and that $ \gamma $ is parametrised by arc-length. Then
$$ \br{\gamma'\br{t}, \N\br{\gamma\br{t}} \times \gamma'\br{t}, \N\br{\gamma\br{t}}} $$
is an orthonormal basis for $ \RR^3 $. Recall that the curvature of $ \gamma $ is $ \vec{\kappa}\br{t} = \gamma''\br{t} $ and that $ \gamma'' $ is orthogonal to $ \gamma' $, because $ \gamma'\br{t} \cdot \gamma'\br{t} = 1 $, so $ \tod{}{t}\br{\gamma''\br{t} \cdot \gamma'\br{t}} = 0 $, so $ 2\gamma''\br{t} \cdot \gamma'\br{t} = 0 $. So
$$ \vec{\kappa}\br{t} = \k_n\br{t}\N + \k_g\br{t}\br{\N \times \gamma'}, $$
where
$$ \k_n\br{t} = \vec{\kappa}\br{t} \cdot \N $$
is the \textbf{normal curvature} of $ \gamma $ at $ t $, and
$$ \k_g\br{t} = \vec{\kappa}\br{t} \cdot \br{\N \times \gamma'} $$
is the \textbf{geodesic curvature} of $ \gamma $ at $ t $, the amount that $ \gamma $ is curving along $ S $. Then $ \gamma $ is a \textbf{geodesic} if the geodesic curvature is zero.

\lecture{25}{Friday}{30/11/18}

\begin{example*}
Let $ S $ be the $ xy $ plane and $ \N = \br{0, 0, 1} $. Consider a curve
$$ \gamma\br{t} = \br{x\br{t}, y\br{t}, 0} $$
parametrised by arc-length, so $ \br{\tod{x}{t}}^2 + \br{\tod{y}{t}}^2 = 1 $. Then the normal curvature is
$$ \k_n = \vec{\kappa} \cdot \N = \br{x'', y'', 0} \cdot \br{0, 0, 1} = 0, $$
and $ \N \times \gamma' = \br{0, 0, 1} \times \br{x', y', 0} = \br{-y', x', 0} $, so the geodesic curvature is
$$ \k_g = \vec{\kappa} \cdot \br{\N \times \gamma'} = \br{x'', y'', 0} \cdot \br{-y', x', 0} = x'y'' - x''y'. $$
Then $ \vec{\kappa} = \k_g\br{\N \times \gamma'} $, so $ \k_g = 0 $ if and only if $ \vec{\kappa} = 0 $, if and only if $ x'' = y'' = 0 $. That is, geodesics in the plane take the form
$$ \br{x\br{t}, y\br{t}} = \br{a_0, b_0} + t\br{a_1, b_1}, $$
which are straight lines.
\end{example*}

\begin{example*}
Let $ S $ be the unit sphere
$$ \cbr{\br{x, y,z} \in \RR^3 \st x^2 + y^2 + z^2 = 1}. $$
Consider a latitude of radius $ r $,
$$ \gamma\br{t} = \br{r\cos \tfrac{t}{r}, r\sin \tfrac{t}{r}, \sqrt{1 - r^2}}, \qquad 0 \le t \le 2\pi r. $$
Then
$$ \gamma'\br{t} = \br{-\sin \tfrac{t}{r}, \cos \tfrac{t}{r}, 0}, \qquad \N\br{\gamma\br{t}} = \gamma\br{t} = \br{r\cos \tfrac{t}{r}, r\sin \tfrac{t}{r}, \sqrt{1 - r^2}}, $$
so
$$ \N \times \gamma' = \br{-\sqrt{1 - r^2}\cos \tfrac{t}{r}, -\sqrt{1 - r^2}\sin \tfrac{t}{r}, r}, \qquad \vec{\kappa}\br{t} = \gamma''\br{t} = \dfrac{1}{r}\br{-\cos \tfrac{t}{r}, -\sin \tfrac{t}{r}, 0}. $$
Thus
\begin{align*}
\k_g
& = \dfrac{1}{r}\br{-\cos \tfrac{t}{r}, -\sin \tfrac{t}{r}, 0} \cdot \br{-\sqrt{1 - r^2}\cos \tfrac{t}{r}, -\sqrt{1 - r^2}\sin \tfrac{t}{r}, r} \\
& = \dfrac{\sqrt{1 - r^2}}{r}\br{\cos^2 \tfrac{t}{r} + \sin^2 \tfrac{t}{r}}
= \dfrac{\sqrt{1 - r^2}}{r}.
\end{align*}
So a latitude is a geodesic if and only if $ r = 1 $, if and only if it is the equator.
\end{example*}

\pagebreak

\begin{proposition}
Local isometries send geodesics to geodesics.
\end{proposition}

\begin{proof}
Let $ \gamma : \sbr{a, b} \to S $ be a geodesic parametrised by arc-length, and $ F : S \to S' $ a local isometry, so
$$
\begin{tikzcd}
\gamma\br{t} \subset S \arrow{r}{F} & \br{F \circ \gamma}\br{t} \subset S' \\
U \subset \RR^2 \arrow{u}{\phi} \arrow{ur}[swap]{\psi = F \circ \phi}
\end{tikzcd}.
$$
From
$$ \vec{\kappa}\br{t} = \k_n\N\br{\gamma\br{t}} + \k_g\br{\N\br{\gamma\br{t}} \times \gamma'\br{t}}, $$
we see that $ \gamma $ is a geodesic if and only if $ \gamma''\br{t} $ is a multiple of $ \N\br{\gamma\br{t}} $, if and only if
$$ \gamma''\br{t} \cdot \phi_u = \gamma''\br{t} \cdot \phi_v = 0, $$
for $ \phi : U \to \RR^2 $ a chart at $ \gamma\br{t} $. Write
$$ \gamma\br{t} = \phi\br{u\br{t}, v\br{t}}, $$
so
$$ \gamma'\br{t} = \phi_uu' + \phi_vv', \qquad \gamma''\br{t} = \br{\phi_{uu}u' + \phi_{uv}v'}u' + \br{\phi_{uv}u' + \phi_{vv}v'}v', $$
so
\begin{align*}
0
& = \gamma'' \cdot \phi_u
= \br{\phi_u \cdot \phi_{uu}}\br{u'}^2 + 2\br{\phi_u \cdot \phi_{uv}}\br{u'v'} + \br{\phi_u \cdot \phi_{vv}}\br{v'}^2 \\
& = \dfrac{1}{2}\dpd{}{u}\br{\phi_u \cdot \phi_u}\br{u'}^2 + \dpd{}{v}\br{\phi_u \cdot \phi_u}\br{u'v'} + \br{\dpd{}{v}\br{\phi_u \cdot \phi_v} - \dfrac{1}{2}\dpd{}{u}\br{\phi_v \cdot \phi_v}}\br{v'}^2 \\
& = \dfrac{1}{2}\g_{11u}\br{u'}^2 + \g_{11v}\br{u'v'} + \br{\g_{12v} - \dfrac{1}{2}\g_{22u}}\br{v'}^2,
\end{align*}
and similarly for $ \gamma'' \cdot \phi_v $. These are determined by $ u $ and $ v $, and the first fundamental form $ \g $. Since $ F $ is an isometry, it preserves $ \g $ with respect to the chart $ \psi = F \circ \phi $, and $ F\br{\gamma\br{t}} = \psi\br{u\br{t}, v\br{t}} $ implies that
$$ \br{F \circ \gamma}'' \cdot \psi_u = \gamma'' \cdot \phi_u = 0, \qquad \br{F \circ \gamma}'' \cdot \psi_v = \gamma'' \cdot \phi_v = 0. $$
\end{proof}

Let $ \phi : U \to S $ be a chart on a regular surface, and let $ \gamma\br{t} = \phi\br{u\br{t}, v\br{t}} $ be a curve on $ S $. Then $ \gamma $ is a geodesic if and only if \footnote{Exercise}
$$ \br{\g_{11}u' + \g_{12}v'}' = \dfrac{1}{2}\br{\g_{11u}\br{u'}^2 + 2\g_{12u}\br{u'v'} + \g_{22u}\br{v'}^2}, $$
and
$$ \br{\g_{21}u' + \g_{22}v'}' = \dfrac{1}{2}\br{\g_{11v}\br{u'}^2 + 2\g_{12v}\br{u'v'} + \g_{22v}\br{v'}^2}. $$
These are the \textbf{geodesic equations}.

\pagebreak

\subsection{Length-minimising curves}

\lecture{26}{Monday}{03/12/18}

Claim that geodesics minimise arc-length, locally. That is, for any small perturbation $ \beta $ of $ \gamma $, $ \L\br{\beta} \ge \L\br{\gamma} $.

\begin{definition}
A \textbf{variation} of $ \gamma : \sbr{0, L} \to S $ is a smooth map
$$ \function{\sbr{0, L} \times \sbr{-\epsilon, \epsilon}}{S}{\br{t, s}}{\gamma_s\br{t}}, $$
such that
$$ \gamma_0\br{t} = \gamma\br{t}, \qquad \gamma_s\br{0} = \gamma\br{0}, \qquad \gamma_s\br{L} = \gamma\br{L}, \qquad t \in \sbr{0, L}, \qquad s \in \sbr{-\epsilon, \epsilon}. $$
\end{definition}

\begin{proposition}
Geodesics minimise arc-length locally. That is, if $ \gamma : \sbr{0, L} \to S $ is a local minimum for arc-length between $ \gamma\br{0} $ and $ \gamma\br{L} $ and is parametrised by arc-length, then $ \gamma $ is a geodesic.
\end{proposition}

\begin{proof}
Let $ \gamma_s\br{t} $ be a variation of $ \gamma $. Then $ s \mapsto \L\br{\gamma_s} $ is minimised at $ s = 0 $. This is a smooth function of $ s $. So
\begin{align*}
0
& = \eval{\dod{}{s}\br{\L\br{\gamma_s}}}_{s = 0}
= \eval{\dod{}{s}\intd{0}{L}{\sqrt{\gamma_s'\br{t} \cdot \gamma_s'\br{t}}}{t}}_{s = 0} \\
& = \intd{0}{L}{\eval{\tod{}{s}\sqrt{\gamma_s'\br{t} \cdot \gamma_s'\br{t}}}_{s = 0}}{t} & \sqrt{\gamma_s'\br{t} \cdot \gamma_s'\br{t}} \ \text{is smooth} \\
& = \intd{0}{L}{\tfrac{\eval{\tod{}{s}\br{\gamma_s'\br{t} \cdot \gamma_s'\br{t}}}_{s = 0}}{\eval{2\sqrt{\gamma_s'\br{t} \cdot \gamma_s'\br{t}}}_{s = 0}}}{t}
= \intd{0}{L}{\eval{\tod{}{s}\br{\gamma_s'\br{t}} \cdot \gamma_s'\br{t}}_{s = 0}}{t} & \abs{\gamma_0'\br{t}} = \abs{\gamma'\br{t}} = 1 \\
& = \intd{0}{L}{\eval{\tod{}{s}\br{\tod{\gamma_s}{t}}}_{s = 0} \cdot \tod{\gamma_0}{t}}{t}
= \intd{0}{L}{\eval{\tod{}{t}\br{\tod{\gamma_s}{s}}_{s = 0}} \cdot \tod{\gamma_0}{t}}{t} & \br{s, t} \mapsto \gamma_s\br{t} \ \text{is smooth} \\
& = \sbr{\eval{\tod{\gamma_s}{s}}_{s = 0} \cdot \tod{\gamma_0}{t}}_0^L - \intd{0}{L}{\eval{\tod{\gamma_s}{s}}_{s = 0} \cdot \tod[2]{\gamma_0}{t}}{t} & \text{integration by parts} \\
& = -\intd{0}{L}{\eval{\tod{\gamma_s}{s}}_{s = 0} \cdot \br{\k_n\br{t}\N\br{\gamma\br{t}} + \k_g\br{t}\br{\N\br{\gamma\br{t}} \times \gamma'\br{t}}}}{t} & \tod{\gamma_s}{s}\br{0} = \tod{\gamma_s}{s}\br{L} = 0 \\
& = \intd{0}{L}{\eval{\tod{\gamma_s}{s}}_{s = 0} \cdot \br{\k_g\br{t}\br{\N\br{\gamma\br{t}} \times \gamma'\br{t}}}}{t} & \N\br{\gamma\br{t}} \perp \T_{\gamma\br{t}} S.
\end{align*}
Let $ g : \sbr{0, L} \to \RR $ be any smooth function such that $ g\br{0} = g\br{L} = 0 $. Can find a variation $ \gamma_s\br{t} $ with
$$ \eval{\dod{\gamma_s}{s}}_{s = 0} = g\br{t}\br{\N\br{\gamma\br{t}} \times \gamma'\br{t}}. $$
Thus we have
$$ \intd{0}{L}{\k_g\br{t}g\br{t}}{t} = 0, $$
and, since $ \k_g $ is continuous as a function of $ t $, and $ g $ is arbitrary, it follows that $ \k_g\br{t} = 0 $ for all $ t $, that is $ \gamma $ was a geodesic.
\end{proof}

\begin{remark*}
\hfill
\begin{itemize}
\item Geodesics need not be global minima for arc-length between two points. In $ \S^2 $ with usual metric, a major arc is a geodesic but not a global minimum for arc-length.
\item Geodesics between two points need not be unique. All great circles are geodesics from $ \N $ to $ S $.
\item Geodesics connecting two points need not exist. In $ \RR^2 \setminus \cbr{0} $, $ 1 $ and $ -1 $ has no geodesic, and $ \RR^2 \setminus \cbr{0} $ is locally isometric to $ \RR^2 $. Local isometries preserve geodesics, so geodesics in $ \RR^2 \setminus \cbr{0} $ are straight lines.
\end{itemize}
In general, it is hard to find geodesics explicitly. Need to solve the geodesic equation.
\end{remark*}

\pagebreak

\section{The Gauss-Bonnet theorem and applications}

\lecture{27}{Tuesday}{04/12/18}

Equates geometry and topology. Will prove this from its local version.

\subsection{Local version of Gauss-Bonnet}

\begin{theorem}[Local Gauss-Bonnet]
Let $ \phi : U \to S $ be a chart which is smooth in $ \overline{U} $, the closure of $ U $ in $ \RR^2 $, such that $ S = \phi\br{\overline{U}} $ is an oriented surface with boundary $ \partial S = \phi\br{\partial\overline{U}} $. Suppose that $ \overline{U} $ is diffeomorphic to a disc. Then,
$$ \intd{S}{}{\K}{A} + \intd{\partial S}{}{\k_g}{s} = 2\pi, $$
where $ A $ is the area, $ s $ is the arc-length, and $ \partial S $ is positively oriented with respect to $ S $. A \textbf{surface with boundary} is the same as a regular surface except that some points $ p \in S $ have charts $ \phi : U \subset \RR^2 \to S $, where $ U $ is the intersection of an open neighbourhood of $ \br{0, 0} $ in $ R^2 $ and $ \cbr{y \ge 0} $. $ \phi $ here is smooth, which means there exists an open set $ V \subset \RR^3 $ and $ \Phi : V \to \RR^3 $ smooth such that $ \phi = \eval{\Phi}_U $ and $ U \subseteq V $. Then $ \partial S $ is the collection of all points with charts like this, a collection of regular curves, and $ \partial S $ is \textbf{positively oriented} if in some parametrisation $ \gamma : \sbr{a, b} \to \partial S $, $ \N \times \gamma' $ points into $ S $.
\end{theorem}

\begin{proof}
The idea is to recall
$$ \intd{S}{}{\K}{A} = \iintd{U}{\br{\K \circ \phi}\abs{\tpd{\phi}{u} \times \tpd{\phi}{v}}}{u}{v}. $$
We will find functions $ M, L : U \to \RR $ such that
$$ \iintd{U}{\br{\K \circ \phi}\abs{\tpd{\phi}{u} \times \tpd{\phi}{v}}}{u}{v} = \iintd{U}{\tpd{M}{u} \times \tpd{L}{v}}{u}{v} = \int_{\partial U} \, L \, \d u + M \, \d v = 2\pi - \intd{\partial U}{}{\k_g}{s}, $$
by applying Green's theorem. For a precise version, make an orthonormal basis for $ \T_p S $. Take
$$ E_1 = \dfrac{\phi_u}{\abs{\phi_u}}, \qquad E_2 = \N \times E_1, $$
so $ \N = E_1 \times E_2 $. Have an orthonormal basis $ E_1 $ and $ E_2 $ for $ \T_p S $ along $ \partial S $. So there exists a continuous function $ \theta : \sbr{0, L} \to \RR $ such that
$$ \gamma'\br{t} = \cos \theta\br{t}E_1\br{\gamma\br{t}} + \sin \theta\br{t}E_2\br{\gamma\br{t}}. $$
Here $ L $ is the length of the boundary, and $ \gamma : \sbr{0, L} \to \partial S $ is an arc-length parametrisation of $ \partial S $.
\begin{itemize}
\item Claim that
$$ \k_g\br{\gamma\br{t}} = \theta'\br{t} - E_1\br{t} \cdot E_2'\br{t}. $$
$ \k_g = \gamma'' \cdot \br{\N \times \gamma'} $. Then
$$ \gamma''\br{t} = \br{-\sin \theta E_1 + \cos \theta E_2}\theta'\br{t} + \cos \theta E_1' + \sin \theta E_2', \quad \N \times \gamma'\br{t} = -\sin \theta E_1 + \cos \theta E_2, $$
so
$$ \k_g\br{\gamma\br{t}} = \gamma''\br{t} \cdot \br{\N \times \gamma'\br{t}} = \theta'\br{t} + \br{\cos \theta E_1' + \sin \theta E_2'} \cdot \br{-\sin \theta E_1 + \cos \theta E_2}. $$
Now,
$$ E_1\br{t} \cdot E_1\br{t} = 1, \qquad E_1\br{t} \cdot E_2\br{t} = 0, \qquad E_2\br{t} \cdot E_2\br{t} = 1. $$
Differentiating,
$$ E_1 \cdot E_1' = 0, \qquad E_1 \cdot E_2' + E_1' \cdot E_2 = 0, \qquad E_2 \cdot E_2' = 0, $$
so $ \k_g = \theta' - E_1 \cdot E_2' $, which was the claim.

\lecture{28}{Friday}{07/12/18}

\item Claim that
$$ \intd{0}{L}{E_1\br{t} \cdot E_2'\br{t}}{t} = \iintd{U}{\br{\K \circ \phi}\abs{\phi_u \times \phi_v}}{u}{v}. $$
This will help because it implies
$$ \intd{S}{}{\K}{A} + \intd{\partial S}{}{\k_g}{s} = \intd{0}{L}{\theta'\br{t}}{t}. $$

\pagebreak

Let us show that
\begin{equation}
\label{eq:4}
\dpd{E_1}{u} \cdot \dpd{E_2}{v} - \dpd{E_1}{v} \cdot \dpd{E_2}{u} = \K\abs{\phi_u \times \phi_v}
\end{equation}
Assuming $ \br{\ref{eq:4}} $ then, setting $ \gamma\br{t} = \phi\br{u\br{t}, v\br{t}} $, we have
\begin{align*}
\intd{0}{L}{E_1\br{t} \cdot E_2'\br{t}}{t}
& = \intd{0}{L}{E_1\br{t} \cdot \br{\tpd{E_2}{u}u' + \tpd{E_2}{v}v'}}{t}
= \int_{\partial U} \, E_1\br{t} \cdot \tpd{E_2}{u} \, \d u + E_1\br{t} \cdot \tpd{E_2}{v} \, \d v \\
& = \iintd{U}{\br{\tpd{}{u}\br{E_1\br{t} \cdot \tpd{E_2}{v}} - \tpd{}{v}\br{E_1\br{t} \cdot \tpd{E_2}{u}}}}{u}{v} \\
& = \iintd{U}{\br{\tpd{E_1}{u} \cdot \tpd{E_2}{v} - \tpd{E_1}{v} \cdot \tpd{E_2}{u}}}{u}{v} = \iintd{U}{\K\abs{\phi_u \times \phi_v}}{u}{v},
\end{align*}
by Green's theorem and $ \br{\ref{eq:4}} $, which proves the claim. It remains to prove $ \br{\ref{eq:4}} $. Note
$$ E_1 \cdot \dpd{E_1}{u} = 0, \qquad E_1 \cdot \dpd{E_2}{u} + \dpd{E_1}{u} \cdot E_2 = 0, \qquad \dots. $$
So
$$ \dpd{E_1}{u} = aE_2 + \br{\dpd{E_1}{u} \cdot \N}\N = aE_2 - \br{E_1 \cdot \dpd{\N}{u}}\N = aE_2 + \A\br{E_1, \dpd{\phi}{u}}N, $$
for some constant $ a $. Similarly,
$$ \dpd{E_1}{v} = bE_2 + \A\br{E_1, \dpd{\phi}{v}}\N, \qquad \dpd{E_2}{u} = cE_1 + \A\br{E_2, \dpd{\phi}{u}}\N, \qquad \dpd{E_2}{v} = dE_1 + \A\br{E_2, \dpd{\phi}{v}}\N, $$
for some constants $ b, c, d $. Thus
$$ \dpd{E_1}{u} \cdot \dpd{E_2}{v} - \dpd{E_1}{v} \cdot \dpd{E_2}{u} = \A\br{E_1, \dpd{\phi}{u}}\A\br{E_2, \dpd{\phi}{v}} - \A\br{E_2, \dpd{\phi}{u}}\A\br{E_1, \dpd{\phi}{v}}. $$
Writing
$$ \dpd{\phi}{u} = c_{11}E_1 + c_{12}E_2, \qquad \dpd{\phi}{v} = c_{21}E_1 + c_{22}E_2, $$
this is
\begin{align*}
\dpd{E_1}{u} \cdot \dpd{E_2}{v} - \dpd{E_1}{v} \cdot \dpd{E_2}{u}
= \ & \br{c_{11}\A\br{E_1, E_1} + c_{12}\A\br{E_1, E_2}} \cdot \br{c_{21}\A\br{E_2, E_1} + c_{22}\A\br{E_2, E_2}} \\
& - \br{c_{11}\A\br{E_2, E_1} + c_{12}\A\br{E_2, E_2}} \cdot \br{c_{21}\A\br{E_1, E_1} + c_{22}\A\br{E_1, E_2}} \\
= \ & \br{c_{11}c_{22} - c_{12}c_{21}} \cdot \br{\A\br{E_1, E_1}\A\br{E_2, E_2} - \A\br{E_1, E_2}\A\br{E_2, E_1}} \\
= \ & \abs{\dpd{\phi}{u} \times \dpd{\phi}{v}}\det \A
= \abs{\dpd{\phi}{u} \times \dpd{\phi}{v}}\K,
\end{align*}
as claimed.
\end{itemize}
So at this point we know
$$ \intd{S}{}{\K}{A} + \intd{\partial S}{}{\k_g}{s} = \intd{0}{L}{\tod{\theta}{t}}{t}. $$
It remains to show that the right hand side is $ 2\pi $. Since
$$ \intd{0}{L}{\tod{\theta}{t}}{t} = \theta\br{L} - \theta\br{0} $$
is the total rotation of $ \gamma'\br{t} $ along the path $ \partial S $ with respect to the frame $ \br{E_1\br{t}, E_2\br{t}} $, the fact that this is $ 2\pi $ follows from the fact that $ \overline{U} $ is only diffeomorphic to a disc, so replace it by a disc. There is a \textbf{regular homotopy}
$$ \function[H]{\sbr{0, 1} \times \sbr{0, 1}}{X}{\br{s, t}}{\gamma_s\br{t}}, $$
smooth in $ s $ and $ t $, from the boundary curve to an approximate tiny circle in $ \T_p S $. Since $ \Ind \gamma = \theta\br{L} - \theta\br{0} $ is invariant under regular homotopy, $ \Ind \gamma $ is the index of a small circle in the plane, which is $ 2\pi $.
\end{proof}

\pagebreak

\subsection{Gauss-Bonnet for curvilinear triangles}

\lecture{29}{Monday}{10/12/18}

\begin{definition}
A \textbf{curvilinear triangle} in $ \RR^2 $ is a continuous map $ \beta : \RR \to \RR^2 $ such that
\begin{itemize}
\item $ \beta\br{t} = \beta\br{t + 3} $ for all $ t $, and
\item there exist $ t_0, t_1, t_2 \in \intco{0, 3} $ such that $ 0 \le t_0 < t_1 < t_2 < 3 $ and
\begin{itemize}
\item $ \beta $ is smooth on $ \br{t_0, t_1} $, $ \br{t_1, t_2} $, and $ \br{t_2, t_3} $ where $ t_3 = t_0 + 3 $,
\item $ \eval{\beta}_{\intco{0, 3}} $ is injective, and
\item $ \beta_-'\br{t_i} $ and $ \beta_+'\br{t_i} $ exist and meet at an angle $ \theta_i $ that is not a multiple of $ \pi $.
\end{itemize}
\end{itemize}
\end{definition}

Let $ T \subset U \subset \RR^2 $ be a curvilinear triangle and let $ \phi : U \to S $ be a chart on a regular surface $ S $. At each vertex of $ \phi\br{T} $, the edges incident to that vertex are parametrised by $ \gamma_{in} : \intoc{-\epsilon, 0} \to S $ and $ \gamma_{out} : \intco{0, \epsilon} \to S $. The tangent vectors meet at some angle $ \theta $ for $ -\pi < \theta < \pi $, so
$$ \cos \theta = \dfrac{\gamma_{in}'\br{0} \cdot \gamma_{out}'\br{0}}{\abs{\gamma_{in}'\br{0}}\abs{\gamma_{out}'\br{0}}}. $$
Choose the sign of $ \theta $ such that $ \theta > 0 $ if $ \br{\gamma_{in}'\br{0}, \gamma_{out}'\br{0}, \N} $ is a positive basis of $ \RR^3 $ and $ \theta < 0 $ if $ \br{\gamma_{in}'\br{0}, \gamma_{out}'\br{0}, \N} $ is a negative basis of $ \RR^3 $. Then $ \theta $ is the \textbf{exterior angle} at the vertex. The \textbf{interior angle} is $ \pi - \theta $.

\begin{theorem}[Gauss-Bonnet for curvilinear triangles]
Let $ T \subset U \subset \RR^2 $ be a curvilinear triangle and let $ \phi : U \to S $ be a chart on the regular surface $ S $. Suppose that $ \phi\br{\partial T} \subset S $ is oriented positively with respect to the orientation of $ S $. Let $ \theta_1, \theta_2, \theta_3 $ be the exterior angles and let $ \gamma_1, \gamma_2, \gamma_3 $ be the edges of $ \phi\br{T} $ parametrised by arc-length. Then,
$$ \sum_{i = 1}^3 \intd{\gamma_i}{}{\k_g}{s} + \sum_{i = 1}^3 \theta_i = 2\pi - \intd{\phi\br{T}}{}{\K}{A}. $$
\end{theorem}

Think of the left hand side as $ \intd{\phi\br{\partial T}}{}{\k_g}{s} $ where $ \phi\br{\partial T} $ is thought as having a $ \delta $-function of curvature of size $ \theta_i $ at the vertex $ i $.

\begin{proof}
Essentially the same as in local Gauss-Bonnet. We still integrate $ \k_g = \theta' - E_1 \cdot E_2' $, apply Green's theorem to the triangle $ T $, and get
$$ \intd{\phi\br{T}}{}{\K}{A} + \sum_{i = 1}^3 \intd{\gamma_i}{}{\k_g}{s} = \intd{\partial S}{}{\theta'\br{t}}{t}. $$
But now, by topology, we have
$$ \intd{\partial S}{}{\theta'\br{t}}{t} = 2\pi - \sum_{i = 1}^3 \theta_i, $$
because the tangent vectors to the $ \gamma_i $ along $ \phi\br{T} = \partial\overline{S} $ still rotate a total of $ 2\pi $, but this includes jumps of $ \theta_1, \theta_2, \theta_3 $ at the three corners, so that $ \intd{\partial S}{}{\theta'\br{t}}{t} $ does not include these jumps.
\end{proof}

The following are consequences. Consider a \textbf{geodesic triangle} on $ S $ of geodesic edges, and interior angles $ \alpha_1, \alpha_2, \alpha_3 $. Gauss-Bonnet says,
$$ \intd{\phi\br{T}}{}{\K}{A} = 2\pi - \sum_{i = 1}^3 \intd{\gamma_i}{}{\k_g}{s} - \sum_{i = 1}^3 \theta_i = 2\pi - \sum_{i = 1}^3 \br{\pi - \alpha_i} = \sum_{i = 1}^3 \alpha_i - \pi. $$
For plane triangles, $ \K = 0 $, so
$$ \sum_{i = 1}^3 \alpha_i = \pi. $$
For geodesic triangles $ T $ on a sphere of radius one, $ \K = 1 $, so
$$ \sum_{i = 1}^3 \alpha_i = \pi + \area T \ge \pi. $$

\pagebreak

\subsection{Gauss-Bonnet theorem}

\begin{proposition}
Every compact regular surface admits a \textbf{triangulation}
$$ S = \bigcup_{i = 1}^N T_i, $$
that is a partition of $ S $ into curvilinear triangles $ T_i $ such that whenever $ T_i \cap T_j $ is non-empty then $ T_i \cap T_j $ is an edge or vertex of both $ T_i $ and $ T_j $, and each edge belongs to
\begin{itemize}
\item exactly two of the $ T_i $ if that edge lies in the interior of $ S $, and
\item in exactly one of the $ T_i $ if that edge lies in $ \partial S $.
\end{itemize}
\end{proposition}

$ S $ is sewn together from curvilinear triangles.

\begin{proof}
By compactness, we can cover $ S $ by finitely many charts $ \phi_i : U_i \to S $ such that $ U_i $ is homeomorphic to a disc. Without loss of generality we may assume
$$ \phi_i\br{U_i} \nsubseteq \bigcup_{j \ne i} \phi_j\br{U_j}, $$
because otherwise just remove $ \br{U_i, \phi_i} $ from our cover. Then
$$ V_i = \phi_i\br{U_i} \setminus \bigcup_{j \ne i} \phi_j\br{U_j} $$
is non-empty, and closed in $ S $. Draw a curve $ C_i \subset \phi_i\br{U_i} $ which bounds a neighbourhood of $ V_i $. The union of the $ C_i $ divides $ S $ into regions, each of which is contained in a single chart, and that chart is homeomorphic to a disc in $ \RR^2 $. Now work in $ U_i $. Now have a closed curve in $ U_i \subset \RR^2 $. Triangulate the interior of this.
\end{proof}

\lecture{30}{Tuesday}{11/12/18}

\begin{lemma}
Three curves with the same vertex of interior angles $ \alpha_1 $ and $ \alpha_2 $ imply that the sum of the interior angles is $ \alpha_1 + \alpha_2 $.
\end{lemma}

\begin{proof}
Without loss of generality we can take the velocity vectors $ \gamma_{in}'\br{0} $ and $ \gamma_{out}'\br{0} $ as unit vectors. Let $ \br{\cos \alpha_1, \sin \alpha_1} $ be a velocity vector of interior angle $ \alpha_1 $, $ \br{\cos \br{\alpha_1 + \alpha_2}, \sin \br{\alpha_1 + \alpha_2}} $ be a velocity vector of interior angle $ \alpha_2 $, and $ \phi $ be the difference of the interior angles. Then
\begin{align*}
\cos \phi
& = \br{\cos \br{\alpha_1 + \alpha_2}, \sin \br{\alpha_1 + \alpha_2}} \cdot \br{\cos \alpha_1, \sin \alpha_1} \\
& = \cos \br{\alpha_1 + \alpha_2}\cos \alpha_1 + \sin \br{\alpha_1 + \alpha_2}\sin \alpha_1
= \cos \br{\br{\alpha_1 + \alpha_2} - \alpha_1}
= \cos \br{\alpha_2},
\end{align*}
so $ \phi = \alpha_2 $.
\end{proof}

What is the \textbf{Euler characteristic}? Take a triangulation of $ S $. Then
$$ \chi\br{S} = V - E + F, $$
where $ V $ is the number of vertices in the triangulation, $ E $ is the number of edges in the triangulation, and $ F $ is the number of triangles in the triangulation. This is independent of the choice of triangulation of $ S $. Follows from the Gauss-Bonnet theorem and can prove this topologically.

\begin{example*}
\hfill
\begin{itemize}
\item Triangulate $ \S^2 $ by an inflated tetrahedron. Then $ V = 4, E = 6, F = 4 $, so $ \chi\br{\S^2} = 4 - 6 + 4 = 2 $.
\item Triangulate $ \T^2 $ by two triangles in a square. Then $ V = 1, E = 3, F = 2 $, so $ \chi\br{\T^2} = 1 - 3 + 2 = 0 $.
\end{itemize}
\end{example*}

\begin{exercise*}
\hfill
\begin{itemize}
\item Triangulate $ \T^2 $ by eight triangles in a square, and compute $ \chi\br{\T^2} $ using this triangulation.
\item Triangulate $ \Sigma_g $, a surface of genus $ g $, by two tori without a disc and $ g - 2 $ tori without two discs, and show $ \chi\br{\Sigma_g} = 2 - 2g $.
\end{itemize}
\end{exercise*}

\pagebreak

\begin{theorem}[Gauss-Bonnet]
Let $ S \subset \RR^3 $ be a compact oriented surface, with, a possibly empty, positively oriented boundary. Then
$$ \intd{S}{}{\K}{A} + \intd{\partial S}{}{\k_g}{s} = 2\pi\chi\br{S}, $$
where $ \chi $ is the Euler characteristic. If $ \partial S = \emptyset $, then
$$ \intd{S}{}{\K}{A} = 2\pi\chi\br{S}. $$
\end{theorem}

\begin{proof}
Triangulate $ S $, so
$$ S = \bigcup_{i = 1}^N T_i, $$
where $ T_i $ is a curvilinear triangle with exterior angles $ \theta_{ij} $ for $ 1 \le j \le 3 $. Set the interior angles $ \alpha_{ij} = \pi - \theta_{ij} $. Then
$$ \intd{T_i}{}{\K}{A} + \intd{\partial T_i}{}{\k_g}{s} = 2\pi - \sum_{j = 1}^3 \theta_{ij} = \sum_{j = 1}^3 \alpha_{ij} - \pi, $$
by Gauss-Bonnet for $ T_i $. The integral of $ \k_g $ along the edge $ T_i \cap T_j $ cancels because it occurs for $ T_i $ and $ T_j $ with opposite signs. All of the interior edges cancel out. At each interior vertex, the interior angles add up to $ 2\pi $. Boundary edges do not cancel. At boundary vertices, the interior angles sum to $ \pi $. Adding everything up,
$$ \sum_{i = 1}^N \intd{T_i}{}{\K}{A} + \sum_{i = 1}^N \intd{\partial T_i}{}{\k_g}{s} = \sum_{i = 1}^N \sum_{j = 1}^3 \alpha_{ij} - \sum_{i = 1}^N \pi, $$
so
$$ \intd{S}{}{\K}{A} + \intd{\partial S}{}{\k_g}{s} = 2\pi V - \pi\#\cbr{\text{boundary vertices}} - \pi F. $$
But counting edges,
$$ 3F = \#\cbr{\text{interior edges}} + \#\cbr{\text{boundary edges}} = 2E - \#\cbr{\text{boundary edges}} = 2E - \#\cbr{\text{boundary vertices}}. $$
So
$$ \intd{S}{}{\K}{A} + \intd{\partial S}{}{\k_g}{s} = 2\pi\br{V + F} - \pi\br{3F + \#\cbr{\text{boundary vertices}}} = 2\pi\br{V - E + F}. $$
\end{proof}

\begin{corollary}
Let $ S \subset \RR^3 $ be a compact oriented surface with $ \partial S = \emptyset $, such that $ \K \ge 0 $ on $ S $. Then $ S \cong \S^2 $.
\end{corollary}

\begin{proof}
There exists some point with $ \K > 0 $ because there exists an elliptic point on $ S $, so
$$ \chi\br{S} = \dfrac{1}{2\pi}\intd{S}{}{\K}{A} > 0. $$
But $ \chi\br{S} = 2 - 2g $ with $ g $ the genus and so $ g = 0 $ because $ \chi\br{S} > 0 $.
\end{proof}

\end{document}